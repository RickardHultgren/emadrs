\documentclass[12pt,a4paper,oneside]{article}
\newcommand{\latex}{\LaTeX\xspace}
\usepackage{textcomp}
\usepackage{tabularx}
\usepackage[table]{xcolor}% http://ctan.org/pkg/xcolo
\usepackage[latin1]{inputenc}
%\usepackage[swedish]{babel}
\usepackage{/usr/lib/R/share/texmf/tex/latex/Sweave}
\usepackage[math]{iwona}
\usepackage[T1]{fontenc}
%\usepackage[swedish]{babel}
\usepackage[UKenglish]{babel}
\usepackage{graphicx}
\usepackage{hyperref}
\usepackage{url} 
\renewcommand{\oldstylenums}[1]{{\fontfamily{pplj}\selectfont #1}}
%\usepackage[textwidth=11cm]{geometry}
%\newcommand{\blob}{\rule[-.2\unitlength]{2\unitlength}{.5\unitlength}}
%\renewcommand{\\_}{\hspace{0.1cm}}
\usepackage{bold-extra}
\usepackage{multirow}
%\urlstyle{same}
%\urlstyle{sf}
\def\mydate{\leavevmode\hbox{\the\year-\twodigits\month-\twodigits\day}}
\def\twodigits#1{\ifnum#1<10 0\fi\the#1}
\usepackage[round,comma]{natbib}
\usepackage{enumitem}
\usepackage{titlesec}
\titleformat*{\section}{\normalsize\bfseries\vspace{0.25cm}}
\titleformat*{\subsection}{\vspace{-0.25cm}\normalsize\it\vspace{0.25cm}}
%\titleformat*{\subsubsection}{\large\bfseries}
%\titleformat*{\paragraph}{\large\bfseries}
%\titleformat*{\subparagraph}{\large\bfseries}

\let\oldcite\cite
\renewcommand*\cite[1]{\textsuperscript{\oldcite{#1}}}

\makeatother
\bibliographystyle{unsrt}
%\bibliographystyle{ieeetr}
%\usepackage{natbib}
\usepackage{cclicenses}
\usepackage{nicefrac}

%\bibliography{test}
%\usepackage[sort, numbers]{natbib}

\begin{document}






leadership\_contact\_patient %'>"leadership\_contact\_patient"</a> from 9 file.
 {\it  assistant\_physician %[ 165: 202] 
} 
Och att patienten och jag är överens. %<br><a href='#leadership\_contact\_patient+b\\ 
 {\it assistant\_physician %[ 362: 445] 
} 
om det är något psykiatriskt då får man känna efter mer. Man får vara mer förljsam. %<br><a href='#leadership\_contact\_patient+b\\ 
 {\it auxiliary\_nurse\_2 %[ 168: 240] 
} 
Man försöker ju göra sitt bästa, men det kan vara frustrerande att t.ex. %<br><a href='#leadership\_contact\_patient+b\\ 
 {\it foot\_therapist %[  11: 739] 
} 
Eller när patienten sitter i stolen hos mig Då börjar dom att prata och det har hänt några gånger också till läkaren till kallats speciellt att de var min mat psykiskt dåligt. Det har till exempel hänt att en patient har uttryckt vilja att gå till det mesta livet så att säga, och då har jag tillkallat läkare. det är alltså inte bara fötter utan mycket prat om annat också. alltså när dom sitter så sitter dom i med fotbad och slappnar av och då börjar de dom prata. Man är lite halvt en dietist då. Det är mycket annat. Ibland känns det som att jag inte kan ge bra svar på frågorna, men ibland kan jag ge bra svar. Ibland vill patienterna veta mer om resultat från undersökningar. Ibland vill de veta mer om diabetes och mat.  %<br><a href='#leadership\_contact\_patient+b\\ 
 {\it nurse\_1 %[  11:  38] 
} 
Tillgodose patientens behov %<br><a href='#leadership\_contact\_patient+b\\ 
 {\it nurse\_1 %[ 183: 263] 
} 
och inte göra patienten alltför förbannad. Försöker linda bort aggressiviteten.<p> %<br><a href='#leadership\_contact\_patient+b\\ 
 {\it nurse\_COPD %[ 238: 479] 
} 
Ibland vill du inte komma på grund av att de inte vill veta att det är så dåligt som det kanske är. Sen så har vi då de som kommer på årskontroller som har sin diagnos och då har vi problem med att de kanske isolerar sig och har det jobbigt. %<br><a href='#leadership\_contact\_patient+b\\ 
 {\it nurse\_COPD %[1096:1144] 
} 
Känna av. Märka vad det är som har hänt just nu. %<br><a href='#leadership\_contact\_patient+b\\ 
 {\it nurse\_DM %[1010:1034] 
} 
Vad är det som har hänt? %<br><a href='#leadership\_contact\_patient+b\\ 
 {\it nurse\_geriatric %[  11: 290] 
} 
Vi har ju allt. Det beror ju på att äldre har inte bara en sjukdom. Det är ju diabetes, KOL. Det är alla saker och vad gäller psykisk ohälsa så är det mest anhöriga som har besvär. Patienter och anhöriga ska kunna kontakta direkt. Oftast är det anhöriga som behöver samtalsstöd.  %<br><a href='#leadership\_contact\_patient+b\\ 
 {\it nurse\_geriatric %[ 554: 580] 
} 
Patientens behov i centrum %<br><a href='#leadership\_contact\_patient+b\\ 
 {\it physician %[ 202: 344] 
} 
Sen får man känna av vad det handlar om till exempel suicidrisk. Visa att vården finns där för sådana besvär, för det är svårt att söka själv. %<br><a href='#leadership\_contact\_patient+b\\ 
 {\it psychotherapist2 %[ 425: 470] 
} 
Det finns alltså en dialog runt behandlingen. %<br><a href='#leadership\_contact\_patient+b'>

%Coding of <a id='
leadership\_empathy %'>"leadership\_empathy"</a> from 1 file.
 {\it administrator1 %[643:649] 
} 
Empati %<br><a href='#leadership\_empathy+b'>
\\
\\
management\_contact\_patient %'>"management\_contact\_patient"</a> from 4 file.
 {\it  administrator1 %[160: 182] 
} 
a och boka in patiente %<br><a href='#management\_contact\_patient+b\\ 
 {\it administrator2 %[ 47:  92] 
} 
Man kan ju få journalkopior och tider av oss. %<br><a href='#management\_contact\_patient+b\\ 
 {\it administrator2 %[229: 306] 
} 
Det är väl att vi har ju mycket tidböcöker -- Vi ska ju kunna erbjuda tider.  %<br><a href='#management\_contact\_patient+b\\ 
 {\it nurse\_1 %[ 56: 107] 
} 
Nu kommer ju telefonsamtalen fram. Det känner vi av %<br><a href='#management\_contact\_patient+b\\ 
 {\it nurse\_1 %[122: 181] 
} 
Sitter mest i telefonrådgivningen, och vill få jobbet gjort %<br><a href='#management\_contact\_patient+b\\ 
 {\it psychotherapist2 %[882:1013] 
} 
<p>De flesta vill träffa samtalsterapeut personligen, men i brist på det går det per telefon. Finns väl formulärtjänst via internet.  %<br><a href='#management\_contact\_patient+b'>
\\
\\

%Coding of <a id='
management\_contact\_staff %'>"management\_contact\_staff"</a> from 1 file.
 {\it foot\_therapist %[844:881] 
} 
Pratar med doktorn om mål i diabetes. %<br><a href='#management\_contact\_staff+b'>
\\
\\
5 Codings of <a id='management\_financial %'>"management\_financial"</a> from 2 file.
 {\it  administrator1 %[528:555] 
} 
ACG ska spegla verkligheten %<br><a href='#management\_financial+b\\ 
 {\it administrator1 %[242:375] 
} 
Ibland kan det vara diagnoskoder som vi letar efter som man inte stöter på ofta. Och vi letar efter kroniska ICD koder sedan tidigare %<br><a href='#management\_financial+b\\ 
 {\it administrator1 %[389:511] 
} 
Det är ju ACG. Med diagnoser ser vi hur vi ligger till. Sedan vi har börjat  koda alla kroniska diagnoser så gick vi upp.  %<br><a href='#management\_financial+b\\ 
 {\it administrator2 %[151:216] 
} 
Vi gör sammanställningar varje vecka för hela verskamhetsområdet. %<br><a href='#management\_financial+b\\ 
 {\it administrator2 %[438:502] 
} 
Man måste ju vara säker på att ICD-koder om nedstämdhet gäller.  %<br><a href='#management\_financial+b'>
\\
\\
3 Codings of <a id='management\_medical\_practice\_patient\_part %'>"management\_medical\_practice\_patient\_part"</a> from 3 file.
 {\it  foot\_therapist %[752: 832] 
} 
Man kan se  resultatet av egenvård,  om t.ex. Om de smörjt i fötterna varje dag. %<br><a href='#management\_medical\_practice\_patient\_part+b\\ 
 {\it nurse\_COPD %[890:1005] 
} 
Sluta röka. Om de slutar så kan de ju faktiskt må bättre. Gångtest kan göras men görs sällan på grund av tidsbrist. %<br><a href='#management\_medical\_practice\_patient\_part+b\\ 
 {\it nurse\_DM %[804: 922] 
} 
Och de har viktnedgångsmål så klart. De har en del krav på sig vad gäller mål för att få fortsätta en viss behandling. %<br><a href='#management\_medical\_practice\_patient\_part+b'>
\\
\\
26 Codings of <a id='management\_medical\_practice\_staff\_part %'>"management\_medical\_practice\_staff\_part"</a> from 9 file.
 {\it  assistant\_physician %[  11:  36] 
} 
Diagnostik och behandling %<br><a href='#management\_medical\_practice\_staff\_part+b\\ 
 {\it assistant\_physician %[  51:  86] 
} 
Labblistor. Används lite som facit. %<br><a href='#management\_medical\_practice\_staff\_part+b\\ 
 {\it assistant\_physician %[  99: 165] 
} 
Oftast symptomlindring, och att värden rätt sida referensvärdena.  %<br><a href='#management\_medical\_practice\_staff\_part+b\\ 
 {\it assistant\_physician %[ 291: 356] 
} 
Om det är somatiskt rätt fram så gör man på samma sätt varje gång %<br><a href='#management\_medical\_practice\_staff\_part+b\\ 
 {\it auxiliary\_nurse\_1 %[  82: 122] 
} 
Jag försöker hjälpa till så gott jag kan %<br><a href='#management\_medical\_practice\_staff\_part+b\\ 
 {\it auxiliary\_nurse\_2 %[  28:  36] 
} 
Hälsan.  %<br><a href='#management\_medical\_practice\_staff\_part+b\\ 
 {\it auxiliary\_nurse\_2 %[  54: 154] 
} 
Vi håller på mycket med sår. Såren minskar ju. De mäter med längd och olika cirklar. Görs månadsvis. %<br><a href='#management\_medical\_practice\_staff\_part+b\\ 
 {\it auxiliary\_nurse\_2 %[ 241: 372] 
} 
Undersöka sår som går fram och tillbaka. Det är inte så mycket som jag kan påverka. Jag gör ju det jag kan efter min bästa förmåga. %<br><a href='#management\_medical\_practice\_staff\_part+b\\ 
 {\it auxiliary\_nurse\_2 %[ 462: 492] 
} 
Det är mitt jobb att fixa det. %<br><a href='#management\_medical\_practice\_staff\_part+b\\ 
 {\it nurse\_COPD %[  11: 237] 
} 
Det är i regel en del i utredningen man gör spirometri. Man har i regel varit hos läkaren och fått reda på att man ska göra spirometri. Ibland säger de att de misstänker KOL. Ibland vet de ju det själva då de har varit rökare. %<br><a href='#management\_medical\_practice\_staff\_part+b\\ 
 {\it nurse\_COPD %[ 480: 815] 
} 
Då får man ju också erbjuda någon att prata mer med för att de ska kunna må bättre. Vi har även ett frågeformulär kring psykosocialt. Det handlar mycket om hur patienten kan klara av sin vardag. Det är ju det som egentligen är behandlingen. Det finns ju egentligen inte så mycket annat att göra än att spara på energin och röra på sig. %<br><a href='#management\_medical\_practice\_staff\_part+b\\ 
 {\it nurse\_COPD %[ 829: 877] 
} 
Det är ju självskattningsformulär och spirometri %<br><a href='#management\_medical\_practice\_staff\_part+b\\ 
 {\it nurse\_DM %[  11: 680] 
} 
Jag måste ju stå i telefonen och göra en bedömning utan att se personen. Jag måste ju lyssna in på kort tid. Det är tidsbegränsat också. Och göra en korrekt bedömning. Var den hör hemma och hur snabbt de måste in. Jag är inte alltid bombarderad av data, men det ringer nån och mår väldigt dåligt och då måste jag veta vad jag ska fråga om. Och det är inte alltid så enkelt Speciellt som jag inte har den specifika utbildningen men det gäller inte diabetes, men även där måste jag veta vilka patienter som kanske behöver gå till annat område och få ett fjärde ben att stå på i diabetes-behandlingen, så därför har vi börjat erbjuda samtalsterapi för diabetes patienter. %<br><a href='#management\_medical\_practice\_staff\_part+b\\ 
 {\it nurse\_DM %[ 693: 751] 
} 
Vid diabetes kan man se förändring i vikt eller blodsocker %<br><a href='#management\_medical\_practice\_staff\_part+b\\ 
 {\it nurse\_DM %[ 773: 803] 
} 
Tydliga mål enligt riktlinjer. %<br><a href='#management\_medical\_practice\_staff\_part+b\\ 
 {\it nurse\_geriatric %[ 302: 359] 
} 
Klocktest, olika frågeformulär, bl.a. skatta hur man mår. %<br><a href='#management\_medical\_practice\_staff\_part+b\\ 
 {\it nurse\_geriatric %[ 371: 464] 
} 
Att få så gott som möjligt för patienterna och se i kvalitetsregister hur vi ligger över lag. %<br><a href='#management\_medical\_practice\_staff\_part+b\\ 
 {\it physician %[  82: 201] 
} 
Det de söker för måste ju åtgärdas. Jag tycker det är oerhört viktigt Se till att uppföljning på den andra biten sker.  %<br><a href='#management\_medical\_practice\_staff\_part+b\\ 
 {\it psychotherapist1 %[  11: 655] 
} 
Det är ju en svår fråga. Jag tänker om vi utgår från samtalsmottagningen här på vårdcentralen, så är det väl mycket bedömning först och främst. Göra bedömningar av patienters psykiska mående och sedan i den bedömningen ingår där ju att göra bedömning om de kan få hjälp - om det är hälso- och sjukvård, för det första. Kan ju också vara någonting som är ganska normalt, vanlig ångest som alla har. Och om det är det, så är det något som ska till primärvården eller om vi ska på remittera vidare. Och sen är det att hålla i behandling när man väl har fått patienter till primärvården. Så tänker man att man kan hjälpa dem må bättre genom samtal. %<br><a href='#management\_medical\_practice\_staff\_part+b\\ 
 {\it psychotherapist1 %[ 667:1025] 
} 
Vi mäter ju med skattningsskalor. Objektivt? Det är ju patienten själv som ska skatta. Ii början av en behandling och sen ser man att symptom sjunker och att de kommit rätt och vi använder Ossr ssr. Och vi skattar patientens tillfredsställelse och tänker att den ska bli bättre. De uppskattar tillfredsställelse på fyra olika områden för att ge oss feedback. %<br><a href='#management\_medical\_practice\_staff\_part+b\\ 
 {\it psychotherapist1 %[1036:1125] 
} 
De brukar ha en tydlighet att de vill klara av saker som att till exempel gå och handla . %<br><a href='#management\_medical\_practice\_staff\_part+b\\ 
 {\it psychotherapist1 %[1217:1443] 
} 
Det beror på hur mycket information sköterskorna har fått eller hur mycket information läkarna har tagit ibland kan det vara svårt för vissa situationer. Ibland kan du ta med dig imorgon förmiddag.<p>Utredande. Känna in känslor. %<br><a href='#management\_medical\_practice\_staff\_part+b\\ 
 {\it psychotherapist2 %[  11:  52] 
} 
Ja du. Symptomlindring. Funktionsökning.  %<br><a href='#management\_medical\_practice\_staff\_part+b\\ 
 {\it psychotherapist2 %[  66: 424] 
} 
Vi mäter ju med formulär. Nästan varje patient fyller före varjesamtal i en symptomskattning. Efter samtalen fyller de en skala som mäter hur nöjda de har varit med innehållet i just det samtalet. Hos många finns också specifika skalor för deras problem. Om man mäter symptom och livskvalitet -- så tar man upp det under samtalet och adresserar förändringar. %<br><a href='#management\_medical\_practice\_staff\_part+b\\ 
 {\it psychotherapist2 %[ 484: 868] 
} 
Dialogen avgör om man har mål eller inte. Patienten kanske tycker det är rimligt att de inte kommer ha stöttning från oss och ska kunna fungera -- Klara av sitt jobb eller familjesituation. Så den är ju ofta ganska tydlig. Vi har ju ganska ofta en tidssatt målbild -- Det där ska vi uppnå inom 8 samtal t.ex. Sen lyckas man ju inte alltid det. Men sen får man se hur långt man kommit. %<br><a href='#management\_medical\_practice\_staff\_part+b\\ 
 {\it psychotherapist2 %[1102:1199] 
} 
Det ingår ju alltid att man bedömer depression. Det gör vi på alla nya patienter. Det är jobbet.  %<br><a href='#management\_medical\_practice\_staff\_part+b'>
\\
\\
management\_medical\_record %'>"management\_medical\_record"</a> from 2 file.
 {\it  administrator1 %[187:201] 
} 
 skriva diktat %<br><a href='#management\_medical\_record+b\\ 
 {\it administrator2 %[306:350] 
} 
Diktaten ska skrivas, och remisser ska iväg. %<br><a href='#management\_medical\_record+b\\ 
 {\it administrator2 %[106:149] 
} 
Vid mäter hur mycket diktat vi ligger efter %<br><a href='#management\_medical\_record+b'>Back<a><br><br>




\end{document}

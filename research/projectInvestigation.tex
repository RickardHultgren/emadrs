\documentclass[english]{beamer}
\usepackage[utf8]{inputenc}
\usepackage{url}
\usepackage[english]{babel}
\usepackage[T1]{fontenc}
\usepackage[math]{iwona}
\setcounter{tocdepth}{1}
\usepackage{xcolor}
\usepackage{mathtools}
\renewcommand{\oldstylenums}[1]{{\fontfamily{pplj}\selectfont #1}}

%\date{October 3--4, 2009}

\usetheme{Boadilla}
\useoutertheme{infolines}
\setbeamerfont{section in head/foot}{family=\sffamily}
\setbeamerfont{subsection in head/foot}{family=\sffamily}
\setbeamerfont{title in head/foot}{family=\sffamily}
\setbeamerfont{author in head/foot}{family=\sffamily}
\setbeamerfont{date in head/foot}{family=\sffamily}
\setbeamerfont{institute in head/foot}{family=\sffamily, size=\tiny}
\setbeamerfont{block title}{family=\sffamily}
\setbeamerfont{block title alerted}{family=\sffamily}
\setbeamerfont{block title example}{family=\sffamily}
%\renewcommand{\familydefault}{\rmdefault}
\usepackage{xcolor}
\definecolor{darkgreen}{HTML}{008800}
\definecolor{lila}{HTML}{aa00aa}

\begin{document}

\title{Personalens attityd till screening och uppföljning av depressionsbehandling via patientens mobiltelefon.}
%\author[ABC]{ABC\\[5mm]{\footnotesize \textbf{Supervisors:}\\Prof. QWERTY\\Prof. DEF GHI\\Prof. Jklmno Pqrst\\}}
\author[]{Första författare: Rickard Hultgren\newline Handledare: Mikael Sandlund\newline Bihandledare: Heljä Pihkala}
%\url{http://slackware.com/~alien/}\\
%\texttt{alien@slackware.com}

\institute{\\
Umeå universitet\\
%\url{http://umu.se/}
}

\begin{frame}
	\titlepage
\end{frame}

\section{Introduction}
\begin{frame}
	\frametitle{}
	\frametitle{}
	\fbox{\parbox{\textwidth}{{\it Sjukvårdslagen;} Målet för hälso- och sjukvården är en god hälsa och en vård på lika villkor för hela befolkningen.}}\\
	De senaste åren har landstingens utgifter ökat med ca. 5\% per år.\\
	Korrigerat för inflation blir det ca. 3\% per år.\\
	Vi har snart inte råd med god hälsa.\\
	\pause
	\textcolor{lila}{Men vilka är villkoren för vården? Låt oss använda modellen SMART och göra en brainstorm på följande tema:\\Vilka villkor ställer vården på dig som del i personalen när det kommer till:}\\
	\begin{itemize}
		\item \textcolor{lila}{Specific:}
		\item\textcolor{lila}{Measurable:}
		\item\textcolor{lila}{Achievable:}
		\item\textcolor{lila}{Relevant:}
		\item\textcolor{lila}{Time-bound:}
	\end{itemize}
\end{frame}


%\section{Introduction}
%\begin{frame}
%\frametitle{Sjukvårdsarbetets villkor}
%\colorbox{black!20}{\parbox{0.45\textwidth}{
%	Traditionella villkor\\\ \\\ 
%	\small
%	\pause
%	\begin{itemize}
%		\item Individuellt bemötande av patienten.\\ 
%		\textcolor{red}{{\bf Failed} Ett vanligt klagomål är dåligt bemötande.}\\\ \\\ \\\ \\
%		\pause
%		\item Individuellt bemötande av patientens behov.\\
%		\textcolor{red}{{\bf Failed} Vården har oftast inte tid att se hur olika behov påverkar varandra.}\\\ \\
%	\end{itemize}
%	\pause
%}}\hspace{1em}\colorbox{black!20}{\parbox{0.45\textwidth}{
%	Med digitala verktyg kan alternativa villkor göras\\
%	\small
%	\pause
%	\begin{itemize}
%		\item Objektivt bedömning av patientens data.\\
%		\textcolor{darkgreen}{Genom att fokusera mer på insamling och automatisk processering av data blir patienten screenad inför ett första möte.}\\
%		\pause
%		\item Ta reda på patientens behov.\\
%		\textcolor{darkgreen}{Genom datainsamling tas patienternas behov in i kalkyleringen. Både landsting och kommuner kan fra nytta av detta.}\\\vspace{.33em}
%	\end{itemize}
%	}}\\
	%\pause
%\end{frame}

\begin{frame}
\frametitle{Kan en digital/elektronisk lösning hjälpa?}
	Vad för grupp av patologier ska en sådan lösning fokusera på?
	%Vilka kriterier har en sådan lösning?\\
	Kriterier för att få så stor effekt som möjligt för så lite nedlagt arbete som möjligt:
	%\pause
	\begin{itemize}
	\item patologin ska vara väldigt vanlig;
	%\pause
	\item som sjukvården är dålig på att upptäcka;
	%\pause
	\item och som är viktig att upptäcka, då den kan vara orsakad av andra patologier;
	%\pause
	\item allvarliga följder om ej behandlad;
	\end{itemize}
	\pause
	Depression borde vara en rimlig patologi att fokusera på.
\end{frame}

\begin{frame}
\textcolor{lila}{Beskriv med ett ord vad du upplever då en patient verkar vara deprimerad, även om hen söker för nåt annat?}
\end{frame}

\begin{frame}
	Punktprevalens för depression anses vara 5\% -- folkhälsoproblem.\\
	%\pause
	Svårt få fram bra statistik för depression: %\pause förrutom vid suicid.
	\begin{itemize}
	\item Stort mörkertal?
	\item Upptäcks ej alltid i sjukvård
	\end{itemize}
	%\pause
	%Svårt få fram bra statistik för depression... %\pause förrutom vid suicid.
	Men enligt statistiken vi kan få fram:
	%\pause
	\begin{itemize}
	\item 50--80\% av suicid anses bero på depression.
	%\pause
	\item Suicid är den ledande dödsorsaken bland män mellan 15 och 44 år i Sverige.
	%\pause
	\item Drygt 2/3 av alla suicidfall uppskattas nyligen ha varit i kontakt med sjukvården. Bara hälften av dessa har kontakt med psykiatrisk klinik.
	\end{itemize}
	\pause
	{\textit{Brister i informationsöverföring mellan patient och sjukvård?}}
%	\pause
%	\textit{Hur kan en app komma till nytta i informationsöverföringen mellan patient och sjukvård?}

\end{frame}

\begin{frame}
Tänk dig att flertalet patienter regelbundet fyller i MADRS-S en gång i veckan...\\\ \\
\pause
\textcolor{lila}{Hur kan MADRS-S-score komma till användning i dina arbetsuppgifter?}\\
\pause
\textcolor{lila}{Vad skulle det betyda för arbetskvaliteten?}\\
\pause
\textcolor{lila}{Vad skulle det betyda för arbetskvantiteten?}\\
\pause
\textcolor{lila}{Hur skulle din arbetssituation förändras?}\\\ \\
\pause
\textit{Kanske bristerna i informationsöverföring mellan patient och vård kan lösas med en mobiltelefonapp?}
\end{frame}


\subsection{}
\begin{frame}
	\frametitle{\center{eMADRS\\\small prototyp-app för android}}
{\center{Digital form av formuläret MADRS-S.\\
\includegraphics[scale=0.4]{emadrs1.png}
\\\ \\
}}
\end{frame}

\begin{frame}
	\frametitle{\center{eMADRS\\\small prototyp-app för android}}
{\center{MADRS-S mäter graden av depressions-symptom.\\
\includegraphics[scale=0.4]{emadrs2.png}
\\\ \\
}}
\end{frame}

\begin{frame}
	\frametitle{\center{eMADRS\\\small prototyp-app för android}}
{\center{MADRS-S ger ej diagnos.\\
\includegraphics[scale=0.4]{emadrs3.png}
\\\ \\
}}
\end{frame}

\begin{frame}
	\frametitle{\center{eMADRS\\\small prototyp-app för android}}
{\center{
\includegraphics[scale=0.4]{emadrs4.png}
\\\ \\
}}
\end{frame}

\begin{frame}
Open source BSD2-licens\\
Källkod:\\
\href{https://github.com/RickardHultgren/emadrs}{\url{https://github.com/RickardHultgren/emadrs}}\\
Nedladdning:\\
\href{https://play.google.com/store/apps/details?id=rickardverner.hultgren.emadrs}{\url{https://play.google.com/store/apps/details?id=rickardverner.hultgren.emadrs}}
\end{frame}

\begin{frame}
\frametitle{Hur kan data från MADRS-S användas?}
	\begin{itemize}
	\item  {Automatisk tidbokning}\\
		\begin{itemize}
		\item  {Nybokning?}
		\item  {Uppföljning?}
		\end{itemize}
	%\pause
	\item  {Inför ett besök kan automatisk bokning av lab-prover göras utifrån riskbedömning för ålder, kön, läkemedel etc. Riskkalkylering kan göras genom programmeringsspråket {\textcolor{blue}{\href{https://github.com/RickardHultgren/lympha}{{\sc lympha}}}} eller kanske Office {\sc{365}}}.\vspace{-1em}
		\begin{itemize}
		\item  {hormonella orsaker, t.ex. hypotyreos}
		\item  {autoimmuna orsaker, t.ex. {\sc sle}}.
		\end{itemize}
	%\pause
	\item  {Om patienten även har möjlighet att samtidigt skicka {\sc madrs-s}-resulatet till kommunen:}\\
		\begin{itemize}
		\item kan det gagna samarbete med kommunala institutioner som t.ex. beroendeenhet
		\end{itemize}
	%\pause	
	%\item  {automatisk tidbokning}\\
	\end{itemize}
	\pause	
{\textcolor{lila}{\center{Hur skulle detta påverka ditt jobb?}\\\ \\}}

\end{frame}



\begin{frame}
\ --Är prototypen för eMADRS användbar för vården?\\
%\pause
\ --Inte som det ser ut nu...\\
%\pause
\ --Men vad behövs mer?\\
\pause
\ --Identitets-skyddad informationsöverföring = plugin kopplat till BankId\\
Så här ser möjligheterna ut för vidareutveckling:
\end{frame}

\begin{frame}
\frametitle{\center{Vidare planer för eMADRS}}
{\small
\parbox{0.35\textwidth}{
Core app\\
\colorbox{black!20}{\parbox{0.35\textwidth}{
	\begin{itemize}
	\item  {
		\fcolorbox{black!20}{black!20}{
		\fcolorbox{black}{black!20}{\textcolor{blue}{emergency call}}}\\
		} ringa-112-funktion\\\ 
	\item	{
		\fcolorbox{black!20}{black!20}{
		\fcolorbox{black}{black!20}{\textcolor{blue}{\includegraphics[scale=0.70]{A.pdf}}}}\\
		} språkval
	\end{itemize}
}}}\hspace{1em}$\xrightarrow[]{\text{data}}$\hspace{1em}\parbox{0.5\textwidth}{
BankId-skyddade plug-ins skickar data till databas i sjukvården\\
\colorbox{black!20}{\parbox{0.5\textwidth}{
	1177 plug-in\\\ \\
	Aktuella funktioner:
	\begin{itemize}
		\item {\bf Tidbokning} via internet
	\end{itemize}
}}\\\ \\
\colorbox{black!20}{\parbox{0.5\textwidth}{
	Plug-in skickar till Office 365 databas eller Citrix eller liknande\\\ \\
	Aktuella funktioner:
	\begin{itemize}
		\item {\bf Uppföljning}; Ta emot SMS data från enbart godkända patienter
	\end{itemize}
}}}}\\

\end{frame}

\begin{frame}
\frametitle{Vad innebär detta i praktiken?}
	\begin{block}{Kontaktorsak} 
	\scriptsize{
	 Två veckors anamnes på MADRS-S mellan 15 (lätt depression) och 25 (måttlig depression).\\
	}
	\end{block}
	Förslag på kriterier för att skicka tidbokning till 1177:
	\begin{itemize}
		\item Patienten fyller i eMADRS under två veckor, minst var tredje dag.
		\item Om resultaten konstant visar över ett vissta värde så görs en tidsbokning automatiskt
	\end{itemize}
	\textcolor{lila}{Hur fungerar detta som:}\\
	\textcolor{lila}{\ --som uppföljning?}\\
	\textcolor{lila}{\ --som nybesök?}
\end{frame}

\begin{frame}
\frametitle{Vad innebär detta i praktiken?}
	\begin{block}{Kontaktorsak} 
	\scriptsize{
	 Uppföljning. 2018-02-15 MADRS-S: 15 (lätt depression).\\
	}
	\end{block}
	Förslag på kriterier för att skicka tidbokning RegionJH Office 365 databasen:
	\begin{itemize}
		\item Bara uppföljningspatienter kommer att tillåtas behörighet till databasen.
	\end{itemize}
	\textcolor{lila}{Hur fungerar detta som:}\\
	\textcolor{lila}{\ --som uppföljning?}\\
\end{frame}

\begin{frame}
\frametitle{SWOT-analys angående eMADRS.}

\textcolor{lila}{Vad fick ni upp ögonen för?}

         \begin{tabular}{|p{.33\textwidth}|p{.33\textwidth}|}
             \hline \bfseries Styrkor &\bfseries Svagheter  \\\hline 
             {\parbox{0.2\textwidth}{
                 \begin{itemize}                    
                     \item \  
                     \item \  
                     \item \               
                     \end{itemize} }}
                     &              {\parbox{0.2\textwidth}{
                 \begin{itemize}                    
                     \item \  
                     \item \  
                     \item \               
                     \end{itemize} }}
 \\\hline
             \hline \bfseries Möjligheter &\bfseries Hot  \\\hline 
             {\parbox{0.2\textwidth}{
                 \begin{itemize}                    
                     \item \  
                     \item \  
                     \item \               
                     \end{itemize} }}
                     &              {\parbox{0.2\textwidth}{
                 \begin{itemize}                    
                     \item \  
                     \item \  
                     \item \               
                     \end{itemize} }}
 \\\hline                     
        \end{tabular}

\end{frame}

\begin{frame}
\frametitle{1 ord}
	\textcolor{lila}{Beskriv med ett vad du tycker om eMADRS.}\\
\end{frame}

\begin{frame}
\frametitle{}
	Tack så mycket för deltagandet!\\
\end{frame}

\end{document}

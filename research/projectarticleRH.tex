\documentclass[12pt,a4paper,oneside]{article}
\newcommand{\latex}{\LaTeX\xspace}
\usepackage{textcomp}
\usepackage{tabularx}
\usepackage[latin1]{inputenc}
%\usepackage[swedish]{babel}
\usepackage[T1]{fontenc}
\usepackage[math]{iwona}
%\usepackage[swedish]{babel}
\usepackage[UKenglish]{babel}
\usepackage{graphicx}
\usepackage{hyperref}
\usepackage{url} 
\renewcommand{\oldstylenums}[1]{{\fontfamily{pplj}\selectfont #1}}
%\usepackage[textwidth=11cm]{geometry}
%\newcommand{\blob}{\rule[-.2\unitlength]{2\unitlength}{.5\unitlength}}
\renewcommand{\_}{\hspace{0.1cm}}
\usepackage{bold-extra}
\usepackage{multirow}
%\urlstyle{same}
%\urlstyle{sf}
\def\mydate{\leavevmode\hbox{\the\year-\twodigits\month-\twodigits\day}}
\def\twodigits#1{\ifnum#1<10 0\fi\the#1}
\usepackage[round,comma]{natbib}

\usepackage{titlesec}
\titleformat*{\section}{\normalsize\bfseries\vspace{0.25cm}}
%\titleformat*{\subsection}{\Large\bfseries}
%\titleformat*{\subsubsection}{\large\bfseries}
%\titleformat*{\paragraph}{\large\bfseries}
%\titleformat*{\subparagraph}{\large\bfseries}

\let\oldcite\cite
\renewcommand*\cite[1]{\textsuperscript{\oldcite{#1}}}

\makeatother
\bibliographystyle{unsrt}
%\bibliographystyle{ieeetr}
%\usepackage{natbib}
\usepackage{cclicenses}
\usepackage{nicefrac}

%\bibliography{test}
%\usepackage[sort, numbers]{natbib}
\begin{document}
\title{
\vspace{-3.9cm}
\resizebox{1\hsize}{!}{{\sc\mydate}\ Project article for {\sc t10} scientific work at the medical programme, Ume� university}
\\\vspace{-.4cm}
\resizebox{1\hsize}{!}{{\fontfamily{iwona}\selectfont\small download address: {\textbf{\emph{\href{https://drive.google.com/open?id=1y0DQy-q_4lnFiUrLyjs9XTB1LWZ3SvbS}{\url{https://drive.google.com/open?id=1y0DQy-q_4lnFiUrLyjs9XTB1LWZ3SvbS}}}}}}}
\\\vspace{-.4cm}
\resizebox{1\hsize}{!}{{\fontfamily{iwona}\selectfont\small\cc This work by Rickard Hultgren is licensed under a Creative Commons Attribution {\sc 3.0} Unported License.}}
\\\vspace{-.3cm}
%{\fontsize{15pt}{15pt}{\fontfamily{ptm}\selectfont{Project title:}}\\\fontsize{18pt}{18pt}{\fontfamily{ptm}\selectfont{Staff attitudes towards follow-up of depression treatment via the patient's mobile phone.}}}\\
%\vspace{0.2cm}\fontfamily{iwona}\selectfont
%\hrule
\vspace{-0.5cm}
\fontsize{12pt}{12pt}{\fontfamily{iwona}\selectfont
}}
%\author{{\small First author} {\small \bf Rickard Hultgren:} {\small \it rihu0003@student.umu.se}\vspace{.25cm}\\
% {\small Inst f klinisk vetenskap/psykiatri; Ume� universitet; 901 85 Ume�}\\
% {\small Supervisor:} {\small \bf Mikael Sandlund} {\small \it mikael.sandlund@umu.se}\\
% {\small Assistant supervisor:} {\small \bf Helj� Pihkala} {\small \it helj�.p@hotmail.com}}
\date{}
\maketitle
\vspace{-0.9cm}
%\hrule
\hrule
\ \\
{\small First author:} {\small \bf Rickard Hultgren} {\small \it rihu0003@student.umu.se}\vspace{.25cm}\\
 {\small Supervisor:} {\small \bf Mikael Sandlund} {\small \it mikael.sandlund@umu.se}\\
 {\small\it Inst f klinisk vetenskap/psykiatri; Ume� universitet; 901 85 Ume�}\vspace{.25cm}\\
 {\small Assistant supervisor:} {\small \bf Helj� Pihkala} {\small \it helj�.p@hotmail.com}\\
 {\small\it Inst f klinisk vetenskap/psykiatri; Ume� universitet; 901 85 Ume�}\vspace{.5cm}\\
{\fontsize{15pt}{15pt}{\fontfamily{ptm}\selectfont{Project title:}}\\\fontsize{18pt}{18pt}{\fontfamily{ptm}\selectfont{ Staff attitudes towards follow-up and screening of depression via the patient's mobile phone.}}}\\
\hrule
\ \\
\fontfamily{iwona}\selectfont

\begin{abstract}
\ \\\vspace{-2em}\ \\
%\emph{
\bfseries{
The health care system is in need of new good and cost-effective tools in order to cope with common diseases as depression. How would the health care be effected if patients would send in mood evalution from a smart phone app? Interviews about that topic were performed with focus groups containing of primary health care staff at Hagfors VC in Sweden. The recordings were analyzed with qualitative content analysis. The project shows XYZ qualitative and quantitative benefits from emadrs as a compliment in the follow-up and screening for depressed patients.
}
\end{abstract}

\section*{Background}
In Sweden, the lifetime prevalence of depression is estimated to be 13.2\% among men and 25.1\% among women\cite{numbers0}. There is a well-established relationship between suicide and mood disorders\cite{numbers1.1}. It has been estimated that 50--80\% of completed suicides are associated with mood disorders\cite{numbers1.1}. Suicide is the leading cause of death among men between the ages of 15 and 44 in Sweden\cite{numbers3.0.1}. Nevertheless, it is estimated that just over \nicefrac{2}{3} of all suicide cases had recently been in touch with the healthcare. Only half of them had contact with a psychiatric clinic\cite{numbers2}. In many cases, the suicide could have been prevented if adequate efforts had been made\cite{numbers1}. Guidelines for the treatment and follow-up of depression exist, but the increase in mental problems among young people poses a major challenge\cite{guide1, regionjh1}.

Thus solving the difficult situation require new ways of dealing with depression. Perhaps cell phones can be used to fight depression? Some cell phone apps have been developed for the purpose of benefiting the health care of depressed patients. The apps could be categorized into two groups depending on what end-user they are meant for. If the end-user is a patient, then the app helps the patient track and understand the symptoms through a mood diary\cite{app1}. If the app is meant to be used by healthcare staff, then the app is constructed around different questionnaires\cite{app2}. Both approaches may result in somewhat better results for the patient, but by focusing on either the patient or the staff a key aspect is neglected. In order for the healthcare staff to help the patient as good and effective as possible, it is necessary to focus on the communication between both parties.

\section*{Purpose}
In order for the healthcare staff to give the depressed patient adequate help, the staff needs adequate information about the patient. In investigations of somatic pathologies, adequate laboratory tests are usually done before an appointment. What if the patient's mood could be measured in a similar way before an appointment? With the purpose of enhancing the communication between the patient and healthcare staff an app prototype for android cell phones (e{\sc madrs}) has been developed by the first author\cite{emadrs1, emadrs2}. The app consists of a {\sc madrs-s} form where the result is sent to a phone number as an {\sc sms} text message. {\sc Madrs-s} is a verified tool commonly used for screening and follow-up of depression\cite{madrs2,madrs3}. It consists of {\sc 9} questions where the patient answers with a rating from {\sc 0} to {\sc 6}. The score is categorized as follows:\\

\begin{tabular}{r|l}
{\bf score} & {\bf severity of depression}\\
\hline{\it 0--6} &  no depression\\
{\it 7--19} & mild depression\\
{\it 20--34} & moderate depression\\
{\it 35--60} & severe depression\\
\end{tabular}\\\\

The research question is: What advantages and disadvantages are identified from a professional clinical perspective, using a digital mood evaluation instrument for depression in screening and follow-up? The aim is also to collect proposals for further development of e{\sc madrs}.

\section*{Materials and Methods}
%Interviews with at least {\sc 2} focus groups, consisting of about {\sc 5--7} primary care unit employees from different staff categories that are involved in the treatment of depression. The staff categories concerned are primarily physicians, nurses and members from the so-called psycho-social teams (e.g. psychologists)\cite{goal1}. Each group will consist of as few staff categories as possible. This would be beneficial for the interviews since it would minimize the risk of hierarchical group dynamics.
Interviews were performed with {\sc 2} focus groups, consisting of {\sc 5} primary care unit employees from different staff categories that are directly or indirectly involved in the treatment of depression at Hagfors VC in Sweden. In order to get a holistic picture of how a primary care unit would be effected by e{\sc madrs}, as many staff categories as possible were interviewed\cite{goal1}. Each group consisted of as few staff categories as possible. This would be beneficial for the interviews since it would minimize the risk of hierarchical group dynamics. The following table is a summary of the group members:\\\\
{\bf group 1}\\
\begin{tabular}{r|l}
Work title & Number of participants\\
\hline
Administrators/secretaries & 0\\
Nurse students & 0\\
Nurses & 0\\
Podiatrists \& foot therapists & 0\\
Physician assistants & 0\\
Physicians & 0\\
Psychotherapists & 0\\
\hline
\end{tabular}\\\ \\
{\bf group 2}\\
\begin{tabular}{r|l}
Work title & Number of participants\\
\hline
Administrators/secretaries & 0\\
Nurse students & 0\\
Nurses & 0\\
Podiatrists \& foot therapists & 0\\
Physician assistants & 0\\
Physicians & 0\\
Psychotherapists & 0\\
\hline
\end{tabular}\\
%The interviews will have the following structure:
%\begin{itemize}
%\item {\bf One word exercise}: Each participant describes handling a depressed patient with one word. 
%\item {\bf Questions}: How would your work situation change if you would get a mood evaluation from almost every of the patient before getting in touch with the patient? What would it mean for the work quality? What would it mean for the work quantity?
%\item {\bf Presentation}: What is the app e{\sc madrs}?
%\item {\bf Presentation}: Problems and solutions that the developer can see.
%\item {\bfseries{\scshape{Swot}} analysis} about e{\sc madrs}.
%\item {\bf One word exercise}: Each participant describes e{\sc madrs} with one word.
%\end{itemize}
The interviews are then analyzed using qualitative content analysis\cite{analysis1}. From the recordings, codes are derived and categorized. The categories are then grouped into themes.

\section*{Results}
From the recordings, the following codes and  were derived and categorized as well as themed:\\
\begin{tabular}{r|l|l}
Code & Category & Theme\\
\hline
Administrators/secretaries & \multirow{ 2}{*}{1} & \multirow{ 3}{*}{1}\\
Nurse students \\
Nurses & 0\\
Podiatrists \& foot therapists & 0 & X\\
\hline
\end{tabular}\\\ \\
The depressed patient is often locked in a state of not being able to manage the worsening of the depression. The faster the staff knows about a deepening of depression, the faster the staff can intervene. The faster the intervention starts, the better is the prognosis. From the patient's perspective, the new ways of communication through a mobile app could be very beneficial. The key question is how the new possibilities should be handled by the healthcare professionals. Currently, the outward patient doesn't have a frequent contact with health care professionals, and the interventions have to be powerful in order to stop the progression of the depression. In those interventions, the beneficence model is often applied and the patient's integrity and feeling of autonomy are often hurt \cite{ethics1}. If, on the other hand, the patient's condition was to be analyzed more frequently, then perhaps it would be possible to handle the depression in a cost effective and better way. The new ways of managing the patient's health could respect the autonomy of the patient. The patient's health could be managed together with the patient\cite{leader1}. 

The project shows XYZ qualitative and quantitative benefits from e{\sc madrs} as a compliment in the follow-up and screening for depressed patients.

\section*{Discussion}
%yAccording to the Swedish law: ''The goal of the health care is a good health and care on equal terms for the entire population''\cite{law}. 
For at least the last 7 years, the county council's expenses have increased by approximately 5\% per year. Adjusted for inflation, it will be approximately 3\% per annum\cite{numbers3.1, numbers3.2}. We soon cannot afford good health. The strategies in healthcare must change. Hopefully, this project can be a step in the right direction. The results may show new ways to make the depression treatment better and more cost-effective.


%\section*{Planned format and language of the essay}
%The work will be a scientific article in English with the Oxford citation and journal style guidelines\cite{style1}. The cited %article and {\fontfamily{pplj}\selectfont{\LaTeX}} will be used for writing the article\cite{style2}.

%\section*{Timetable}
%The main parts of the project and assigned duration in weeks are listed below: 
%\begin{tabular}{|r|p{2cm}|}
%\hline
%Preparation of the project plan and creation of focus groups. & {{6}}\\
%Collecting data. & {{2}}\\
%Data processing. & {{6}}\\
%Project report. & {{3}}\\
%Preparing for the presentation. & {{2}}\\
%\hline
%\end{tabular}\\

\section*{Ethics}
The project's character is developmental work within the clinic. Therefore it is being examined in terms of confidentiality and safety by the Head of Operations. The project does not fall under the Ethics Testing Act's research definition.

\begin{thebibliography}{11}
\bibitem{numbers0}Kendler KS, Gatz M, Gardner CO, Pedersen NL\emph{A Swedish national twin study of lifetime major depression.}; Am J Psychiatry. 2006 Jan; 163(1):109-14.\\\textbf{\emph{\href{https://www.ncbi.nlm.nih.gov/pubmed/16390897/}{\url{https://www.ncbi.nlm.nih.gov/pubmed/16390897/}}}}
\bibitem{numbers1} \emph{Utv�rdering 2013 -- v�rd och insatser vid depression, �ngest och schizofreni. Indikatorer och underlag f�r bed�mningar.}; Socialstyrelsen\\\textbf{\emph{\href{http://www.socialstyrelsen.se/publikationer2013/2013-6-7}{\url{http://www.socialstyrelsen.se/publikationer2013/2013-6-7}}}}
\bibitem{numbers1.1} Kasper S1, Schindler S, Neumeister A.; \emph{Risk of suicide in depression and its implication for psychopharmacological treatment.}; Int Clin Psychopharmacol. 1996 Jun;11(2):71-9.\\\textbf{\emph{\href{https://www.ncbi.nlm.nih.gov/pubmed/8803644}{\url{https://www.ncbi.nlm.nih.gov/pubmed/8803644}}}}
\bibitem{numbers2} \emph{Sj�lvmord i anslutning till v�rd Socialstyrelsen};\\\textbf{\emph{\href{http://www.socialstyrelsen.se/patientsakerhet/riskomraden/suicid}{\url{http://www.socialstyrelsen.se/patientsakerhet/riskomraden/suicid
}}}}
\bibitem{numbers3.0.1}\emph{Statistics on causes of death 2015 - Socialstyrelsen}; Socialstyrelsen;
\\\textbf{\emph{\href{https://www.socialstyrelsen.se/Lists/Artikelkatalog/Attachments/20291/2016-8-4.pdf}{\url{
https://www.socialstyrelsen.se/Lists/Artikelkatalog/Attachments/20291/2016-8-4.pdf}}}}
\bibitem{numbers3.1} \emph{Resultatr�kning f�r landsting �r 2010--2014}; SCB;\\\textbf{\emph{\href{ http://www.scb.se/hitta-statistik/statistik-efter-amne/offentlig-ekonomi/finanser-for-den-kommunala-sektorn/rakenskapssammandrag-for-kommuner-och-landsting/pong/tabell-och-diagram/kommun--och-landstingssektorn-2014/resultatrakning-for-landsting-ar-20102014/}{\url{http://www.scb.se/hitta-statistik/statistik-efter-amne/offentlig-ekonomi/finanser-for-den-kommunala-sektorn/rakenskapssammandrag-for-kommuner-och-landsting/pong/tabell-och-diagram/kommun--och-landstingssektorn-2014/resultatrakning-for-landsting-ar-20102014/}}}}
\bibitem{numbers3.2} \emph{Resultatr�kning f�r landsting �r 2012--2016}; SCB;\\\textbf{\emph{\href{
http://www.scb.se/hitta-statistik/statistik-efter-amne/offentlig-ekonomi/finanser-for-den-kommunala-sektorn/rakenskapssammandrag-for-kommuner-och-landsting/pong/tabell-och-diagram/kommun--och-landstingssektorn-2016/resultatrakning-for-landsting-ar-2012-2016/}{\url{http://www.scb.se/hitta-statistik/statistik-efter-amne/offentlig-ekonomi/finanser-for-den-kommunala-sektorn/rakenskapssammandrag-for-kommuner-och-landsting/pong/tabell-och-diagram/kommun--och-landstingssektorn-2016/resultatrakning-for-landsting-ar-2012-2016/}}}}
\bibitem{app1}\emph{Appen Uppskatta}; Google Play\\\textbf{\emph{\href https://play.google.com/store/apps/details?id=com.akerlund.uppskattadindag}{\url{{https://play.google.com/store/apps/details?id=com.akerlund.uppskattadindag}}}}
\bibitem{app2}\emph{Appen PsykTools}; Google Play\\\textbf{\emph{\href https://play.google.com/store/apps/details?id=no.sonat.honos}{\url{{https://play.google.com/store/apps/details?id=no.sonat.honos}}}}
%\bibitem{madrs1}Zimmerman, Chelminski, Posternak \emph{The Montgomery Asberg Depression Rating Scale in bipolar II and unipolar out-patients: a 405-patient case study.}; Int Clin Psychopharmacol. 2004 Jan;19(1):1-7; \\\textbf{\emph{\href{http://www.ncbi.nlm.nih.gov/pubmed/15101563}{\url{http://www.ncbi.nlm.nih.gov/pubmed/15101563}}}}
\bibitem{madrs2}Svanborg, P; �sberg, M; \emph{A comparison between the Beck Depression Inventory (BDI) and the self-rating version of the Montgomery �sberg Depression Rating Scale (MADRS)}; J. Affective Disorders. 64 (2-3): 203--216. doi:10.1016/S0165-0327(00)00242-1.\\\textbf{\emph{\href{https://www.ncbi.nlm.nih.gov/pubmed/11313087}{\url{https://www.ncbi.nlm.nih.gov/pubmed/11313087}}}}
\bibitem{madrs3}\emph{Tolkning av MADRS-S}; Region J�nk�pings l�n\\\textbf{\emph{\href{http://plus.rjl.se/info_files/infosida39803/madrs_s_tolkning.pdf}{\url{http://plus.rjl.se/info_files/infosida39803/madrs_s_tolkning.pdf}}}}
\bibitem{emadrs1}Rickard Hultgren; \emph{eMADRS source code}; github.com; \\\textbf{\emph{\href{https://github.com/RickardHultgren/emadrs}{\url{https://github.com/RickardHultgren/emadrs}}}}
\bibitem{emadrs2}Rickard Hultgren; \emph{eMADRS compiled}; play.google.com; \\\textbf{\emph{\href{https://play.google.com/store/apps/details?id=rickardverner.hultgren.emadrs}{\url{https://play.google.com/store/apps/details?id=rickardverner.hultgren.emadrs}}}}
%\bibitem{goal0}Birgitta Lindelius; \emph{�ppna j�mf�relser och utv�rdering 2010 -- Psykiatrisk v�rd}; socialstyrelsen.se; \\\textbf{\emph{\href{http://www.socialstyrelsen.se/publikationer2010/2010-6-6}{\url{http://www.socialstyrelsen.se/publikationer2010/2010-6-6}}}}
\bibitem{goal1}Tracy R.G. Gladstone, William R. Beardslee, Erin E. O'Connor; \emph{The Prevention of Adolescent Depression}; Psychiatr Clin North Am. 2011 Mar; 34(1): 35--52. \\\textbf{\emph{\href{https://www.ncbi.nlm.nih.gov/pmc/articles/PMC3072710/}{\url{https://www.ncbi.nlm.nih.gov/pmc/articles/PMC3072710/}}}}
%\bibitem{goal2}Maryann Davis, Michael T. Abrams, Lawrence S. Wissow, Eric P. Slade; \emph{Identifying young adults at risk of Medicaid enrollment lapses after inpatient mental health treatment}; Psychiatr Serv. 2014 Apr 1; 65(4): 461--468.doi:  10.1176/appi.ps.201300199 \\\textbf{\emph{\href{https://www.ncbi.nlm.nih.gov/pmc/articles/PMC3972275/}{\url{https://www.ncbi.nlm.nih.gov/pmc/articles/PMC3972275/}}}}
\bibitem{guide1}Riitta Sorsa; \emph{Nationella riktlinjer -- M�lniv�er -- V�rd vid depression och �ngestsyndrom -- M�lniv�er f�r indikatorer}; socialstyrelsen.se december 2017\\\textbf{\emph{\href{http://www.socialstyrelsen.se/publikationer2017/2017-12-1}{\url{http://www.socialstyrelsen.se/publikationer2017/2017-12-1}}}}
\bibitem{regionjh1}Majvor Enstr�m; \emph{Granskning av Psykiatrin 2014 Region J�mtland-H�rjedalen}\\\textbf{\emph{\href{https://www.regionjh.se/download/18.61342ea415bcfb51720c5fd7}{\url{https://www.regionjh.se/download/18.61342ea415bcfb51720c5fd7}}}}
%\bibitem{style0}Vakhtang Tchantchaleishvili, Jan D Schmitto; \emph{Preparing a scientific manuscript in Linux: Today's possibilities and limitations.}; BMC Res Notes. 2011; 4: 434. \\\textbf{\emph{\href{https://www.ncbi.nlm.nih.gov/pmc/articles/PMC3227619/}{\url{https://www.ncbi.nlm.nih.gov/pmc/articles/PMC3227619/}}}}
\bibitem{style1}\emph{Style Guide for Authors} \\\textbf{\emph{\href{https://academic.oup.com/cdj/pages/Style_Guide}{\url{https://academic.oup.com/cdj/pages/Style_Guide}}}}
\bibitem{style2}Barbara J. Hoogenboom, Robert C. Manske; \emph{How to write a scientific article.} Int J Sports Phys Ther. 2012 Oct; 7(5): 512--517. \\\textbf{\emph{\href{https://www.ncbi.nlm.nih.gov/pmc/articles/PMC3474301/}{\url{https://www.ncbi.nlm.nih.gov/pmc/articles/PMC3474301/}}}}
%\bibitem{style3}\emph{Latex Instructions - Elsevier}; Elsevier 2018;\\\textbf{\emph{\href{https://www.elsevier.com/authors/author-schemas/latex-instructions}{\url{https://www.elsevier.com/authors/author-schemas/latex-instructions}}}}
\bibitem{leader1}Kotter JP \emph{What leaders really do.} Harvard Business Review 1990
\bibitem{ethics1}Will JF \emph{A brief historical and theoretical perspective on patient autonomy and medical decision making: Part I: The beneficence model.} Chest. 2011 Mar;139(3):669-673. doi: 10.1378/chest.10-2532. \\\textbf{\emph{\href{https://www.ncbi.nlm.nih.gov/pubmed/21362653}{\url{https://www.ncbi.nlm.nih.gov/pubmed/21362653}}}}

\bibitem{analysis1}Granheim, Lundman \emph{Qualitative content analysis in nursing research concepts, procedures and measures to achieve trustworthiness.} Nurse Educ Today. 2004 Feb;24(2):105-12 \\\textbf{\emph{\href{https://www.ncbi.nlm.nih.gov/pubmed/14769454}{\url{https://www.ncbi.nlm.nih.gov/pubmed/14769454}}}}

%\bibitem{law}H�lso- och sjukv�rdslag (1982:763)\\\textbf{\emph{\href{http://www.notisum.se/rnp/sls/lag/19820763.htm}{\url{http://www.notisum.se/rnp/sls/lag/19820763.htm}}}}


\end{thebibliography} 
 
\end{document}



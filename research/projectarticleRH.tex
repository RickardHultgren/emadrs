\documentclass[12pt,a4paper,oneside]{article}
\newcommand{\latex}{\LaTeX\xspace}
\usepackage{textcomp}
\usepackage{tabularx}
\usepackage[table]{xcolor}% http://ctan.org/pkg/xcolo
\usepackage[latin1]{inputenc}
%\usepackage[swedish]{babel}
\usepackage{/usr/lib/R/share/texmf/tex/latex/Sweave}
\usepackage[math]{iwona}
\usepackage[T1]{fontenc}
%\usepackage[swedish]{babel}
\usepackage[UKenglish]{babel}
\usepackage{graphicx}
\usepackage{hyperref}
\usepackage{url} 
\renewcommand{\oldstylenums}[1]{{\fontfamily{pplj}\selectfont #1}}
%\usepackage[textwidth=11cm]{geometry}
%\newcommand{\blob}{\rule[-.2\unitlength]{2\unitlength}{.5\unitlength}}
%\renewcommand{\_}{\hspace{0.1cm}}
\usepackage{bold-extra}
\usepackage{multirow}
%\urlstyle{same}
%\urlstyle{sf}
\def\mydate{\leavevmode\hbox{\the\year-\twodigits\month-\twodigits\day}}
\def\twodigits#1{\ifnum#1<10 0\fi\the#1}
\usepackage[round,comma]{natbib}
\usepackage{enumitem}
\usepackage{titlesec}
\titleformat*{\section}{\normalsize\bfseries\vspace{0.25cm}}
\titleformat*{\subsection}{\vspace{-0.25cm}\normalsize\it\vspace{0.25cm}}
%\titleformat*{\subsubsection}{\large\bfseries}
%\titleformat*{\paragraph}{\large\bfseries}
%\titleformat*{\subparagraph}{\large\bfseries}

\let\oldcite\cite
\renewcommand*\cite[1]{\textsuperscript{\oldcite{#1}}}

\makeatother
\bibliographystyle{unsrt}
%\bibliographystyle{ieeetr}
%\usepackage{natbib}
\usepackage{cclicenses}
\usepackage{nicefrac}

%\bibliography{test}
%\usepackage[sort, numbers]{natbib}
\begin{document}
\title{
\vspace{-3.9cm}
\resizebox{1\hsize}{!}{{\sc\mydate}\ Project article for {\sc t10} scientific work at the medical programme, Ume\r{a} university}
\\\vspace{-.4cm}
\resizebox{1\hsize}{!}{\fontfamily{iwona}\selectfont\small Download English version from: {\textbf\emph{\href{https://github.com/RickardHultgren/emadrs/blob/master/research/projectarticleRH.pdf}{\url{https://github.com/RickardHultgren/emadrs/blob/master/research/projectarticleRH.pdf}}}}}
\\\vspace{-.4cm}
\resizebox{1\hsize}{!}{\fontfamily{iwona}\selectfont\small Ladda ner svensk version fr\r{a}n: {\textbf\emph{\href{https://github.com/RickardHultgren/emadrs/blob/master/research/projectarticleRHsv.pdf}{\url{https://github.com/RickardHultgren/emadrs/blob/master/research/projectarticleRHsv.pdf}}}}}\\\vspace{-.4cm}
\resizebox{1\hsize}{!}{{\fontfamily{iwona}\selectfont\small\cc This work by Rickard Hultgren is licensed under a Creative Commons Attribution {\sc 4.0} Unported License.}}
\\\vspace{-.3cm}
%{\fontsize{15pt}{15pt}{\fontfamily{ptm}\selectfont{Project title:}}\\\fontsize{18pt}{18pt}{\fontfamily{ptm}\selectfont{Staff attitudes towards follow-up of depression treatment via the patient's mobile phone.}}}\\
%\vspace{0.2cm}\fontfamily{iwona}\selectfont
%\hrule
\vspace{-0.5cm}
\fontsize{12pt}{12pt}{\fontfamily{iwona}\selectfont
}}
%\author{{\small First author} {\small \bf Rickard Hultgren:} {\small \it rihu0003@student.umu.se}\vspace{.25cm}\\
% {\small Inst f klinisk vetenskap/psykiatri; Ume\r{a} universitet; 901 85 Ume\r{a}}\\
% {\small Supervisor:} {\small \bf Mikael Sandlund} {\small \it mikael.sandlund@umu.se}\\
% {\small Assistant supervisor:} {\small \bf Helj{\"a} Pihkala} {\small \it helj{\"a}.p@hotmail.com}}
\date{}
\maketitle
\vspace{-0.9cm}
%\hrule
\hrule
\ \\
{\small First author:} {\small \bf Rickard Hultgren} {\small \it rihu0003@student.umu.se}\vspace{.25cm}\\
 {\small Supervisor:} {\small \bf Mikael Sandlund} {\small \it mikael.sandlund@umu.se}\\
 {\small\it Inst f klinisk vetenskap/psykiatri; Ume\r{a} universitet; 901 85 Ume\r{a}}\vspace{.25cm}\\
% {\small Assistant supervisor:} {\small \bf Helj{\"a} Pihkala} {\small \it helj{\"a}.p@hotmail.com}\\
 {\small Assistant supervisor:} {\small \bf Helj{\"a} Pihkala} {\small \it helja.pihkala@umu.se}\\
 {\small\it Inst f klinisk vetenskap/psykiatri; Ume\r{a} universitet; 901 85 Ume\r{a}}\vspace{.5cm}\\
{\fontsize{15pt}{15pt}{\fontfamily{ptm}\selectfont{Project title:}}\\\fontsize{18pt}{18pt}{\fontfamily{ptm}\selectfont{ Staff attitudes towards follow-up and screening via the patient's smartphone, exemplified by a questionnaire for self-rating of depression symptoms.}}}\\
\hrule
\ \\
\fontfamily{iwona}\selectfont

\begin{abstract}
\ \\\vspace{-2em}\ \\
%\emph{
\bfseries{
The health care system is in need of new cost-effective tools. How would the health care be affected if the primary care units would receive questionnaire results from the patient's smartphone? Interviews on this topic were performed with focus groups containing primary health care staff at Hagfors Primary Care Centre in Sweden. The recordings were examined using qualitative content analysis. The project shows that digital questionnaires has potentials in screening and follow up.
}
\end{abstract}

\section*{Background}
In Sweden, the lifetime prevalence of depression is estimated to be {\sc 13.2}\% among men and {\sc 25.1}\% among women\cite{numbers0}. There is a well-established correlation between suicide and mood disorders\cite{numbers1.1}. It has been estimated that {\sc 50--80}\% of completed suicides are associated with mood disorders\cite{numbers1.1}. Suicide is the leading cause of death among men between the ages of {\sc 15} and {\sc 44} in Sweden\cite{numbers3.0.1}. Nevertheless, it is estimated that just over \nicefrac{2}{3} of all suicide cases had recently been in touch with the healthcare. Only \nicefrac{1}{3} of all suicide cases had contact with a psychiatric clinic\cite{numbers2}. In many cases, the suicide could have been prevented if adequate efforts had been made\cite{numbers1}. Guidelines for the treatment and follow-up of depression exist, but the increase in mental problems among young people poses a major challenge\cite{guide1, regionjh1}.

Thus solving the difficult situation require new ways of dealing with depression. Some smartphone apps have been developed for the purpose of benefiting the health care of depressed patients. The apps could be categorized into two groups depending on what end-user they are meant for. If the end-user is a patient, then the app helps the patient to track and understand the symptoms through a mood diary\cite{app1}. If the app is meant to be used by healthcare staff, then the app is constructed around different questionnaires\cite{app2}. Both approaches may result in somewhat better results for the patient, but by focusing on either the patient or the staff a key aspect is neglected. In order for the healthcare staff to help the patient as good and effective as possible, it is necessary to focus on the communication between both parties.

\section*{Purpose}
In order for the healthcare staff to give the depressed patient adequate help, the staff needs adequate information about the patient. In investigations of somatic pathologies, adequate laboratory tests are usually done before an appointment. What if the patient's mood could be measured in a similar way before an appointment? With the purpose of enhancing the communication between the patient and healthcare staff an app prototype (e{\sc madrs}) for android smartphones has been developed by the first author\cite{emadrs1, emadrs2}. The app consists of a {\sc madrs-s} form. The result is sent as an {\sc sms} text message to a phone number, preferred by the patient. {\sc Madrs-s} is a verified tool commonly used for screening and follow-up of depression\cite{madrs2,madrs3}. It consists of nine questions. The patient answers each question with a rating from zero to six. The total score from all questions is categorized as follows:\\

\begin{tabular}{r|l}
{\bf score} & {\bf severity of depression}\\
\hline{\it 0--6} &  no depression\\
{\it 7--19} & mild depression\\
{\it 20--34} & moderate depression\\
{\it 35--54} & severe depression\\
\end{tabular}\vspace{1em}
\\The research-topics are three areas, closely bound up with each other:
\begin{itemize}
\item[$\alpha$] What advantages and disadvantages are identified from a professional clinical perspective, using a digital evaluation instrument for depression in screening and follow-up?
\item[$\beta$] The aim is also to collect proposals for further development of e{\sc madrs}.
\item[$\gamma$] What staff categories would be most affected by digital questionnaires?
\end{itemize}

\section*{Materials and Methods}
%Interviews with at least {\sc 2} focus groups, consisting of about {\sc 5--7} primary care unit employees from different staff categories that are involved in the treatment of depression. The staff categories concerned are primarily physicians, nurses and members from the so-called psycho-social teams (e.g. psychologists)\cite{goal1}. Each group will consist of as few staff categories as possible. This would be beneficial for the interviews since it would minimize the risk of hierarchical group dynamics.
In order to get a holistic picture of how a primary care unit would be affected by digital questionnaires, many staff categories were interviewed\cite{goal1}. 
Two focus groups were formed, consisting of seven respectively six primary care unit employees from different staff categories that are directly or indirectly involved in the treatment of depression at Hagfors Primary Care Centre in Sweden. The following table is a compilation of the group members:\\\\
\begin{tabular}{p{3em}|p{10em}|l|l}
Group & Work title & Interview 1 & Interview 2 \\
\hline
\multirow{ 5}{*}{A} & Administrators & 1 & 1 \\
& Nurses & 3 & 3 \\
& Foot therapists & 1 & 0\\
& Physicians & 0 & 1\\
& Psychotherapists & 1 & 1\\
\hline
\multirow{ 6}{*}{B} & Administrators & 1 & 1 \\
& Auxiliary nurses & 1 & 2 \\
& Nurses & 0 & 1 \\
& Foot therapists & 0 & 0\\
& Physician assistants & 1 & 1\\
& Psychotherapists & 1 & 1\\
\hline
\end{tabular}\\ \\\ \\\ 
Two 30-minutes long interviews were performed with each group. During the interviews the following questions were discussed:\\\ \\
Interview {\sc 1}\vspace{-.33em}
\begin{enumerate}[label=\sc 1.\arabic*.]
\item {\bf} What is specific, measurable and achievable in your work?
\end{enumerate}
Interview {\sc 2}\vspace{-.33em}
\begin{enumerate}[label=\sc 2.\arabic*.]
%\setlength\itemsep{0em}
\item {\bf} Describe what you experience your work situation when a patient's major issue is not related to depression, but the patient seems to be in a very sad mood? \vspace{-.33em}
\item {\bf} Scenarious are discussed:\vspace{-.33em}
\begin{itemize}\vspace{-.33em}
%\setlength\itemsep{0em}
\item {\bf} What if e{\sc madrs} only could be used for follow-up?\vspace{-.33em}
\item {\bf} What if e{\sc madrs} could be used by everyone to send you mood evaluations?\vspace{-.33em}
\item {\bf} What if the result of e{\sc madrs} automatically could regulate what lab-tests should be performed?
\end{itemize}
\end{enumerate}
%The interviews will have the following structure:
%\begin{itemize}
%\item {\bf One word exercise}: Each participant describes handling a depressed patient with one word. 
%\item {\bf Questions}: How would your work situation change if you would get a mood evaluation from almost every of the patient before getting in touch with the patient? What would it mean for the work quality? What would it mean for the work quantity?
%\item {\bf Presentation}: What is the app e{\sc madrs}?
%\item {\bf Presentation}: Problems and solutions that the developer can see.
%\item {\bfseries{\scshape{Swot}} analysis} about e{\sc madrs}.
%\item {\bf One word exercise}: Each participant describes e{\sc madrs} with one word.
%\end{itemize}
The interviews were then analysed using qualitative content analysis\cite{analysis1}. From the recordings, causation codes were derived and categorized. The work was done with the help of the programming language library {\sc rqda}\cite{rqda}.

\section*{Results}
%\subsection*{Research topic $\alpha$ and $\beta$}
{\bf Research topic $\alpha$ and $\beta$:} Around the example e{\sc madrs} the following potentials, strengths and weaknesses were identified:
\begin{itemize}
\item E{\sc madrs} could be very useful for following up patients that are at risk of relapse of depression.\\{\it Code name in appendix: {\bfseries emadrs\_in\_dev\_only\_follow\_up}}
\item There is a need for digital tools with validated questionnaires for a broader spectrum of pathologies.\\{\it Code name in appendix: {\bfseries emadrs\_in\_dev\_screening\_follow\_up}}
\item These questionnaires should be connected with each other in a controlled way.\\{\it Code name in appendix: {\bfseries emadrs\_in\_dev\_controll}}
\item E{\sc madrs} could reduce the work load of the administrators\\{\it Code name in appendix: {\bfseries emadrs\_already\_(+)\_less\_paper\_work}}
\item Important that someone should be responsible and accountable for the incoming questionnaire results.\\{\it Code name in appendix: {\bfseries emadrs\_already\_(+)\_possebility\_to\_check}}
\item E{\sc madrs} should not be used in the process of diagnosing depression.\\{\it Code name in appendix: {\bfseries emadrs\_not\_in\_dev\_everybody\_diagnostic}}
\item E{\sc madrs} should not be possible to use by everyone in order to send the results to the healthcare provider.\\{\it Code name in appendix: {\bfseries emadrs\_not\_in\_dev\_everybody\_too\_many}}
\end{itemize}
%There should be some kind of flow control for those questionnaires.
{\bf Research topic $\gamma$:} In effective organisations the management is distinguished from the leadership\cite{leader1}. Below is a table describing what staff categories according to the interviews, are managing or leading in their work:
\nopagebreak
\begin{table}
\begin{tabularx}{\textwidth}{|X|X|}
\hline
Tasks of contacting the patient, that imply leadership
{\newline\tiny Coding in appindix: {leadership\_contact\_patient}}&{\begin{itemize}
\vspace{-1.5em}\setlength\itemsep{0em}
{ \item assistant physician}
{ \item auxiliary nurse}
{ \item foot therapist}
{ \item nurse}
{ \item nurse COPD}
{ \item nurse DM2}
{ \item nurse geriatric}
{ \item physician}
{ \item psychotherapist}\vspace{-.5em}\vspace{-.5em}\end{itemize}}\\
\hline
Tasks of contacting the patient, that imply management
{\newline\tiny Coding in appindix: {management\_contact\_patient}}&{\begin{itemize}
\vspace{-1.5em}\setlength\itemsep{0em}
{ \item administrator}
{ \item nurse}
{ \item psychotherapist}
\vspace{-.5em}\end{itemize}}\\
\hline
Tasks of feeling empathy, that imply leadership
{\newline\tiny Coding in appindix: {leadership\_empathy}}&{\begin{itemize}
\vspace{-1.5em}\setlength\itemsep{0em}
{ \item administrator}
\vspace{-.5em}\end{itemize}}\\
\hline
Tasks of contacting staff, that imply management
{\newline\tiny Coding in appindix: {management\_contact\_staff}}&{\begin{itemize}
\vspace{-1.5em}\setlength\itemsep{0em}
{ \item foot\_therapist}
\vspace{-.5em}\end{itemize}}\\
\hline
Managing finances
{\newline\tiny Coding in appindix: {management\_financial}}&{\begin{itemize}
\vspace{-1.5em}\setlength\itemsep{0em}
{ \item administrator}
\vspace{-.5em}\end{itemize}}\\
\hline
Tasks of maintaing patient's health, that imply managing the patient
{\newline\tiny Coding in appindix:\newline {management\_medical\_practice\_patient\_part}}&{\begin{itemize}
\vspace{-1.5em}\setlength\itemsep{0em}
{ \item foot therapist}
{ \item nurse COPD}
{ \item nurse DM2}
\vspace{-.5em}\end{itemize}}\\
\hline
Tasks of maintaining patient's health, that imply managing other professionals
{\newline\tiny Coding in appindix: {management\_medical\_practice\_staff\_part}}&{\begin{itemize}
\vspace{-1.5em}\setlength\itemsep{0em}
{ \item assistant physician}
{ \item auxiliary nurse }
{ \item nurse COPD}
{ \item nurse DM2}
{ \item nurse geriatric}
{ \item physician}
{ \item psychotherapist}
\vspace{-.5em}\end{itemize}}\\
\hline
Tasks of managing medical records
{\newline\tiny Coding in appindix: {management\_medical\_record}}&{\begin{itemize}
\vspace{-1.5em}\setlength\itemsep{0em}
{ \item administrator}
\vspace{-.5em}\end{itemize}}\\
\hline
\end{tabularx}
\end{table}
\newpage
Digital questionnaires are a way to {\it manage the communication with the patient}. According to the table above, the staff categories that are working with that task are: nurses, administrators and psychotherapists. Thus those staff categories would probably be most affected by the digital questionaires.
%The project shows XYZ qualitative and quantitative benefits from e{\sc madrs} as a compliment in the follow-up and screening for depressed patients.
%The depressed patient is often locked in a state of not being able to manage the worsening of the depression. The faster the staff knows about a deepening of depression, the faster the staff can intervene. The faster the intervention starts, the better is the prognosis. From the patient's perspective, the new ways of communication through a mobile app could be very beneficial. The key question is how the new possibilities should be handled by the healthcare professionals. Currently, the outward patient doesn't have a frequent contact with health care professionals, and the interventions have to be powerful in order to stop the progression of the depression. In those interventions, the beneficence model is often applied and the patient's integrity and feeling of autonomy are often hurt \cite{ethics1}. If, on the other hand, the patient's condition was to be analysed more frequently, then perhaps it would be possible to handle the depression in a cost effective and better way. The new ways of managing the patient's health could respect the autonomy of the patient. The patient's health could be managed together with the patient\cite{leader1}. 



\section*{Discussion}
%yAccording to the Swedish law: ''The goal of the health care is a good health and care on equal terms for the entire population''\cite{law}. 
For at least the last seven years, the county councils' expenses have increased by approximately {\sc 5}\% per year. Adjusted for inflation, it will be approximately {\sc 3}\% per annum\cite{numbers3.1, numbers3.2}. The strategies in healthcare must change. Hopefully, this project can be a step in the right direction. The results show new ways to improve the communication between the healthcare system and the patient. 
In order to foresee shifts in work load that the new digital tools could lead to, further scientific work has to be done. This is important since economic benefits could only be gained if the work tasks would change, since the finances are a mirroring of what work has been done.

%\section*{Planned format and language of the essay}
%The work will be a scientific article in English with the Oxford citation and journal style guidelines\cite{style1}. The cited %article and {\fontfamily{pplj}\selectfont{\LaTeX}} will be used for writing the article\cite{style2}.

%\section*{Timetable}
%The main parts of the project and assigned duration in weeks are listed below: 
%\begin{tabular}{|r|p{2cm}|}
%\hline
%Preparation of the project plan and creation of focus groups. & {{6}}\\
%Collecting data. & {{2}}\\
%Data processing. & {{6}}\\
%Project report. & {{3}}\\
%Preparing for the presentation. & {{2}}\\
%\hline
%\end{tabular}\\

\section*{Ethics}
The project's character is developmental work within the clinic. 
Therefore, the project has been approved in terms of confidentiality and safety by the Head of Operations at Hagfors Primary Care Centre in Sweden. The project does not fall under the Ethics Testing Act's research definition.

\begin{thebibliography}{11}
\bibitem{numbers0}Kendler KS, Gatz M, Gardner CO, Pedersen NL\emph{A Swedish national twin study of lifetime major depression.}; Am J Psychiatry. 2006 Jan; 163(1):109-14.\\\textbf{\emph{\href{https://www.ncbi.nlm.nih.gov/pubmed/16390897/}{\url{https://www.ncbi.nlm.nih.gov/pubmed/16390897/}}}}
\bibitem{numbers1} \emph{Utv{\"a}rdering 2013 -- v\r{a}rd och insatser vid depression, \r{a}ngest och schizofreni. Indikatorer och underlag f{\"o}r bed{\"o}mningar.}; Socialstyrelsen\\\textbf{\emph{\href{http://www.socialstyrelsen.se/publikationer2013/2013-6-7}{\url{http://www.socialstyrelsen.se/publikationer2013/2013-6-7}}}}
\bibitem{numbers1.1} Kasper S1, Schindler S, Neumeister A.; \emph{Risk of suicide in depression and its implication for psychopharmacological treatment.}; Int Clin Psychopharmacol. 1996 Jun;11(2):71-9.\\\textbf{\emph{\href{https://www.ncbi.nlm.nih.gov/pubmed/8803644}{\url{https://www.ncbi.nlm.nih.gov/pubmed/8803644}}}}
\bibitem{numbers2} \emph{Sj{\"a}lvmord i anslutning till v\r{a}rd Socialstyrelsen};\\\textbf{\emph{\href{http://www.socialstyrelsen.se/patientsakerhet/riskomraden/suicid}{\url{http://www.socialstyrelsen.se/patientsakerhet/riskomraden/suicid
}}}}
\bibitem{numbers3.0.1}\emph{Statistics on causes of death 2015 - Socialstyrelsen}; Socialstyrelsen;
\\\textbf{\emph{\href{https://www.socialstyrelsen.se/Lists/Artikelkatalog/Attachments/20291/2016-8-4.pdf}{\url{
https://www.socialstyrelsen.se/Lists/Artikelkatalog/Attachments/20291/2016-8-4.pdf}}}}
\bibitem{numbers3.1} \emph{Resultatr{\"a}kning f{\"o}r landsting \r{a}r 2010--2014}; SCB;\\\textbf{\emph{\href{ http://www.scb.se/hitta-statistik/statistik-efter-amne/offentlig-ekonomi/finanser-for-den-kommunala-sektorn/rakenskapssammandrag-for-kommuner-och-landsting/pong/tabell-och-diagram/kommun--och-landstingssektorn-2014/resultatrakning-for-landsting-ar-20102014/}{\url{http://www.scb.se/hitta-statistik/statistik-efter-amne/offentlig-ekonomi/finanser-for-den-kommunala-sektorn/rakenskapssammandrag-for-kommuner-och-landsting/pong/tabell-och-diagram/kommun--och-landstingssektorn-2014/resultatrakning-for-landsting-ar-20102014/}}}}
\bibitem{numbers3.2} \emph{Resultatr{\"a}kning f{\"o}r landsting \r{a}r 2012--2016}; SCB;\\\textbf{\emph{\href{
http://www.scb.se/hitta-statistik/statistik-efter-amne/offentlig-ekonomi/finanser-for-den-kommunala-sektorn/rakenskapssammandrag-for-kommuner-och-landsting/pong/tabell-och-diagram/kommun--och-landstingssektorn-2016/resultatrakning-for-landsting-ar-2012-2016/}{\url{http://www.scb.se/hitta-statistik/statistik-efter-amne/offentlig-ekonomi/finanser-for-den-kommunala-sektorn/rakenskapssammandrag-for-kommuner-och-landsting/pong/tabell-och-diagram/kommun--och-landstingssektorn-2016/resultatrakning-for-landsting-ar-2012-2016/}}}}
\bibitem{app1}\emph{Appen Uppskatta}; Google Play\\\textbf{\emph{\href https://play.google.com/store/apps/details?id=com.akerlund.uppskattadindag}{\url{{https://play.google.com/store/apps/details?id=com.akerlund.uppskattadindag}}}}
\bibitem{app2}\emph{Appen PsykTools}; Google Play\\\textbf{\emph{\href https://play.google.com/store/apps/details?id=no.sonat.honos}{\url{{https://play.google.com/store/apps/details?id=no.sonat.honos}}}}
%\bibitem{madrs1}Zimmerman, Chelminski, Posternak \emph{The Montgomery Asberg Depression Rating Scale in bipolar II and unipolar out-patients: a 405-patient case study.}; Int Clin Psychopharmacol. 2004 Jan;19(1):1-7; \\\textbf{\emph{\href{http://www.ncbi.nlm.nih.gov/pubmed/15101563}{\url{http://www.ncbi.nlm.nih.gov/pubmed/15101563}}}}
%\bibitem{madrs2}Svanborg, P; \r{A}sberg, M; \emph{A comparison between the Beck Depression Inventory (BDI) and the self-rating version of the Montgomery {\"A}\r{A}sberg Depression Rating Scale (MADRS)}; J. Affective Disorders. 64 (2-3): 203--216. doi:10.1016/S0165-0327(00)00242-1.\\\textbf{\emph{\href{https://www.ncbi.nlm.nih.gov/pubmed/11313087}{\url{https://www.ncbi.nlm.nih.gov/pubmed/11313087}}}}
\bibitem{madrs2}Svanborg, P; \r{A}sberg, M; \emph{A comparison between the Beck Depression Inventory (BDI) and the self-rating version of the Montgomery \r{A}sberg Depression Rating Scale (MADRS)}; J. Affective Disorders. 64 (2-3): 203--216. doi:10.1016/S0165-0327(00)00242-1.\\\textbf{\emph{\href{https://www.ncbi.nlm.nih.gov/pubmed/11313087}{\url{https://www.ncbi.nlm.nih.gov/pubmed/11313087}}}}
\bibitem{madrs3}\emph{Tolkning av MADRS-S}; Region J{\"o}nk{\"o}pings l{\"a}n\\\textbf{\emph{\href{http://plus.rjl.se/info_files/infosida39803/madrs_s_tolkning.pdf}{\url{http://plus.rjl.se/info_files/infosida39803/madrs_s_tolkning.pdf}}}}
\bibitem{emadrs1}Rickard Hultgren; \emph{eMADRS source code}; github.com; \\\textbf{\emph{\href{https://github.com/RickardHultgren/emadrs}{\url{https://github.com/RickardHultgren/emadrs}}}}
\bibitem{emadrs2}Rickard Hultgren; \emph{eMADRS compiled}; play.google.com; \\\textbf{\emph{\href{https://play.google.com/store/apps/details?id=rickardverner.hultgren.emadrs}{\url{https://play.google.com/store/apps/details?id=rickardverner.hultgren.emadrs}}}}
%\bibitem{goal0}Birgitta Lindelius; \emph{{\"O}ppna j{\"a}mf{\"o}relser och utv{\"a}rdering 2010 -- Psykiatrisk v\r{a}rd}; socialstyrelsen.se; \\\textbf{\emph{\href{http://www.socialstyrelsen.se/publikationer2010/2010-6-6}{\url{http://www.socialstyrelsen.se/publikationer2010/2010-6-6}}}}
\bibitem{goal1}Tracy R.G. Gladstone, William R. Beardslee, Erin E. O'Connor; \emph{The Prevention of Adolescent Depression}; Psychiatr Clin North Am. 2011 Mar; 34(1): 35--52. \\\textbf{\emph{\href{https://www.ncbi.nlm.nih.gov/pmc/articles/PMC3072710/}{\url{https://www.ncbi.nlm.nih.gov/pmc/articles/PMC3072710/}}}}
%\bibitem{goal2}Maryann Davis, Michael T. Abrams, Lawrence S. Wissow, Eric P. Slade; \emph{Identifying young adults at risk of Medicaid enrollment lapses after inpatient mental health treatment}; Psychiatr Serv. 2014 Apr 1; 65(4): 461--468.doi:  10.1176/appi.ps.201300199 \\\textbf{\emph{\href{https://www.ncbi.nlm.nih.gov/pmc/articles/PMC3972275/}{\url{https://www.ncbi.nlm.nih.gov/pmc/articles/PMC3972275/}}}}
\bibitem{guide1}Riitta Sorsa; \emph{Nationella riktlinjer -- M\r{a}lniv\r{a}er -- V\r{a}rd vid depression och \r{a}ngestsyndrom -- M\r{a}lniv\r{a}er f{\"o}r indikatorer}; socialstyrelsen.se december 2017\\\textbf{\emph{\href{http://www.socialstyrelsen.se/publikationer2017/2017-12-1}{\url{http://www.socialstyrelsen.se/publikationer2017/2017-12-1}}}}
\bibitem{regionjh1}Majvor Enstr{\"o}m; \emph{Granskning av Psykiatrin 2014 Region J{\"a}mtland-H{\"a}rjedalen}\\\textbf{\emph{\href{https://www.regionjh.se/download/18.61342ea415bcfb51720c5fd7}{\url{https://www.regionjh.se/download/18.61342ea415bcfb51720c5fd7}}}}
%\bibitem{style0}Vakhtang Tchantchaleishvili, Jan D Schmitto; \emph{Preparing a scientific manuscript in Linux: Today's possibilities and limitations.}; BMC Res Notes. 2011; 4: 434. \\\textbf{\emph{\href{https://www.ncbi.nlm.nih.gov/pmc/articles/PMC3227619/}{\url{https://www.ncbi.nlm.nih.gov/pmc/articles/PMC3227619/}}}}
%\bibitem{style1}\emph{Style Guide for Authors} \\\textbf{\emph{\href{https://academic.oup.com/cdj/pages/Style_Guide}{\url{https://academic.oup.com/cdj/pages/Style_Guide}}}}
%\bibitem{style2}Barbara J. Hoogenboom, Robert C. Manske; \emph{How to write a scientific article.} Int J Sports Phys Ther. 2012 Oct; 7(5): 512--517. \\\textbf{\emph{\href{https://www.ncbi.nlm.nih.gov/pmc/articles/PMC3474301/}{\url{https://www.ncbi.nlm.nih.gov/pmc/articles/PMC3474301/}}}}
%\bibitem{style3}\emph{Latex Instructions - Elsevier}; Elsevier 2018;\\\textbf{\emph{\href{https://www.elsevier.com/authors/author-schemas/latex-instructions}{\url{https://www.elsevier.com/authors/author-schemas/latex-instructions}}}}
\bibitem{leader1}Kotter JP \emph{What leaders really do.} Harvard Business Review 1990
%\bibitem{ethics1}Will JF \emph{A brief historical and theoretical perspective on patient autonomy and medical decision making: Part I: The beneficence model.} Chest. 2011 Mar;139(3):669-673. doi: 10.1378/chest.10-2532. \\\textbf{\emph{\href{https://www.ncbi.nlm.nih.gov/pubmed/21362653}{\url{https://www.ncbi.nlm.nih.gov/pubmed/21362653}}}}

\bibitem{analysis1}Granheim, Lundman \emph{Qualitative content analysis in nursing research concepts, procedures and measures to achieve trustworthiness.} Nurse Educ Today. 2004 Feb;24(2):105-12 \\\textbf{\emph{\href{https://www.ncbi.nlm.nih.gov/pubmed/14769454}{\url{https://www.ncbi.nlm.nih.gov/pubmed/14769454}}}}

%\bibitem{law}H{\"a}lso- och sjukv\r{a}rdslag (1982:763)\\\textbf{\emph{\href{http://www.notisum.se/rnp/sls/lag/19820763.htm}{\url{http://www.notisum.se/rnp/sls/lag/19820763.htm}}}}

\bibitem{rqda}Ronggui Huang; \\\textbf{\emph{\href{http://rqda.r-forge.r-project.org/}{\url{http://rqda.r-forge.r-project.org/}}}}

\end{thebibliography} 
 
\end{document}

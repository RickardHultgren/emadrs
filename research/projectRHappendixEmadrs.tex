\documentclass[12pt,a4paper,oneside]{article}
\newcommand{\latex}{\LaTeX\xspace}
\usepackage{textcomp}
\usepackage{tabularx}
\usepackage[table]{xcolor}% http://ctan.org/pkg/xcolo
\usepackage[latin1]{inputenc}
%\usepackage[swedish]{babel}
\usepackage{/usr/lib/R/share/texmf/tex/latex/Sweave}
\usepackage[math]{iwona}
\usepackage[T1]{fontenc}
%\usepackage[swedish]{babel}
\usepackage[UKenglish]{babel}
\usepackage{graphicx}
\usepackage{hyperref}
\usepackage{url} 
\renewcommand{\oldstylenums}[1]{{\fontfamily{pplj}\selectfont #1}}
%\usepackage[textwidth=11cm]{geometry}
%\newcommand{\blob}{\rule[-.2\unitlength]{2\unitlength}{.5\unitlength}}
%\renewcommand{\\_}{\hspace{0.1cm}}
\usepackage{bold-extra}
\usepackage{multirow}
%\urlstyle{same}
%\urlstyle{sf}
\def\mydate{\leavevmode\hbox{\the\year-\twodigits\month-\twodigits\day}}
\def\twodigits#1{\ifnum#1<10 0\fi\the#1}
\usepackage[round,comma]{natbib}
\usepackage{enumitem}
\usepackage{titlesec}
\titleformat*{\section}{\normalsize\bfseries\vspace{0.25cm}}
\titleformat*{\subsection}{\vspace{-0.25cm}\normalsize\it\vspace{0.25cm}}
%\titleformat*{\subsubsection}{\large\bfseries}
%\titleformat*{\paragraph}{\large\bfseries}
%\titleformat*{\subparagraph}{\large\bfseries}

\let\oldcite\cite
\renewcommand*\cite[1]{\textsuperscript{\oldcite{#1}}}

\makeatother
\bibliographystyle{unsrt}
%\bibliographystyle{ieeetr}
%\usepackage{natbib}
\usepackage{cclicenses}
\usepackage{nicefrac}

%\bibliography{test}
%\usepackage[sort, numbers]{natbib}

\begin{document}





1 
%Coding of <a id='
emadrs\_already\_(+)\_less\_paper\_work %'>"emadrs\_already\_(+)\_less\_paper\_work"</a> from 1 file.
 {\it administrator2 %[517:590] 
 } 
Vi skannar ju nu massa papper så att få MADRS-S elektroniskt blir lättare %<br><a href='#emadrs\_already\_(+)\_less\_paper\_work+b'>
\\
\\
1 
%Coding of <a id='
emadrs\_already\_(+)\_possebility\_to\_check %'>"emadrs\_already\_(+)\_possebility\_to\_check"</a> from 1 file.
 {\it auxiliary\_nurse\_1 %[161:275] 
 } 
 känns viktigt att kolla upp MADRS -- jag får uppmärksamma någon annan som på , detingår ju inte i mina uppgifter. %<br><a href='#emadrs\_already\_(+)\_possebility\_to\_check+b'>
\\
\\
8 Codings of <a id='emadrs\_in\_dev\_controll %'>"emadrs\_in\_dev\_controll"</a> from 4 file.
 {\it administrator1 %[ 665:1013] 
 } 
Med samtalsmottagningen diskuterade vi igår efter intervjun att istället för att använda MADRS  kan man kanske skicka frågorna som sköterskorna ska ställa. För det tar annars lång tid för sköterskorna. Om sköterskorna istället kan skicka frågor till patientens mobiltelefon. Så kan man sedan under telefonupprgningen hjälpa till på ett annat sätt.  %<br><a href='#emadrs\_in\_dev\_controll+b'>Back
 {\it nurse\_DM %[1053:1119] 
 } 
Bra använda formulär som mäter hur bra patienten mår i allmänhet.  %<br><a href='#emadrs\_in\_dev\_controll+b'>Back
 {\it nurse\_DM %[1208:1248] 
 } 
Jo viktgt veta vad som inte är diabetes. %<br><a href='#emadrs\_in\_dev\_controll+b'>Back
 {\it physician %[ 719: 938] 
 } 
 Syftet ska vara att hitta psykisk ohälsa, så att en kontakt görs i vården. Den stora frågan är hur kontakten skagöras. Du ska fundera på om MADRS är det rätta formuläret. Men tänket är helt rätt, det kan få genomslag.  %<br><a href='#emadrs\_in\_dev\_controll+b'>Back
 {\it physician %[ 522: 651] 
 } 
Jag är mer intresserad av att använda formulär i allmän screening för att korta gapet till att väl söka för sin psykiska ohälsa.  %<br><a href='#emadrs\_in\_dev\_controll+b'>Back
 {\it physician %[ 972:1054] 
 } 
Viktigt ha i åtanke vilka formulär är för vårdpersonal för att validera patienten. %<br><a href='#emadrs\_in\_dev\_controll+b'>Back
 {\it psychotherapist1 %[1644:1748] 
 } 
Låter bra med automatiserade lab prover, men bättre med bättre journalsystem som skulle signalera saker. %<br><a href='#emadrs\_in\_dev\_controll+b'>Back
 {\it psychotherapist1 %[1463:1572] 
 } 
Beror på vad man gör med värdet. Det är viktigt att någon handhar det och möter symptomen. Att det följs upp. %<br><a href='#emadrs\_in\_dev\_controll+b'>
\\
\\
1 
%Coding of <a id='
emadrs\_in\_dev\_only\_follow\_up %'>"emadrs\_in\_dev\_only\_follow\_up"</a> from 1 file.
 {\it nurse\_DM %[1120:1207] 
 } 
Bra att det begränsas till uppföljning så det inte blir som med receptionens blodtryck. %<br><a href='#emadrs\_in\_dev\_only\_follow\_up+b'>
\\
\\
5 Codings of <a id='emadrs\_in\_dev\_screening\_follow\_up %'>"emadrs\_in\_dev\_screening\_follow\_up"</a> from 4 file.
 {\it assistant\_physician %[ 535: 604] 
 } 
MADRS-S blir ju bra screeningverktyg, men kan vara stöd vid återbesök %<br><a href='#emadrs\_in\_dev\_screening\_follow\_up+b'>Back
 {\it nurse\_COPD %[1159:1191] 
 } 
Viktgt veta vad som inte är KOL. %<br><a href='#emadrs\_in\_dev\_screening\_follow\_up+b'>Back
 {\it nurse\_geriatric %[ 598: 623] 
 } 
Allt hör till geriatrik.  %<br><a href='#emadrs\_in\_dev\_screening\_follow\_up+b'>Back
 {\it psychotherapist2 %[1343:1810] 
 } 
Man behöver ha dialog med patienen, men om man avslutar en pateitnet som är rimligt symptomfri så finns det alltid en risk för återfall. Så om man fyller i appen med jämna mellanrum och kommer på annat besök så så kan man kankse se om det stuckit ivväg. Men man måste presentera MADRS bra grafiskt för personalen, så det inte missas. För annars kan det vara farligt eftersom patienten litar på att personalen har sett resultatet. Det skulle nog kunna öka kvaliteten.  %<br><a href='#emadrs\_in\_dev\_screening\_follow\_up+b'>Back
 {\it psychotherapist2 %[1216:1342] 
 } 
Under förutsättningen att appen bara är för personer som behandlas för nedsämdhet, så är det ett jätte bra sätt att följa upp. %<br><a href='#emadrs\_in\_dev\_screening\_follow\_up+b'>
\\
\\
5 Codings of <a id='emadrs\_not\_in\_dev\_everybody\_diagnostic %'>"emadrs\_not\_in\_dev\_everybody\_diagnostic"</a> from 4 file.
 {\it administrator2 %[ 592: 687] 
 } 
Automatisk provtagning: Labprover är inget vi tar ställning till. Vi bara beställer labbprover. %<br><a href='#emadrs\_not\_in\_dev\_everybody\_diagnostic+b'>Back
 {\it physician %[ 652: 720] 
 } 
Men vi ska passa oss för att i övrigt automatisera diagnostisering.  %<br><a href='#emadrs\_not\_in\_dev\_everybody\_diagnostic+b'>Back
 {\it physician %[1055:1135] 
 } 
Automatisera lab-tagning kommer i framtiden, men det tror jag är för stort steg. %<br><a href='#emadrs\_not\_in\_dev\_everybody\_diagnostic+b'>Back
 {\it psychotherapist1 %[1573:1643] 
 } 
Automatisk diagnostisering med blockad kan vara förknippat med risker. %<br><a href='#emadrs\_not\_in\_dev\_everybody\_diagnostic+b'>Back
 {\it psychotherapist2 %[1957:2125] 
 } 
Man ska ju aldrig använda MADRS diagnostiskt för det finns ju de som underrapporterar. Så att använda appen till alla för att söka ny tid kan det vara väldigt farligt.<p> %<br><a href='#emadrs\_not\_in\_dev\_everybody\_diagnostic+b'>
\\
\\
2 Codings of <a id='emadrs\_not\_in\_dev\_everybody\_too\_many %'>"emadrs\_not\_in\_dev\_everybody\_too\_many"</a> from 2 file.
 {\it auxiliary\_nurse\_2 %[669:724] 
 } 
Om man riktar appen till alla blir det för mycket folk. %<br><a href='#emadrs\_not\_in\_dev\_everybody\_too\_many+b'>Back
 {\it physician %[361:521] 
 } 
Risk för bara belastning om det inte bara sker som uppföljning. Jag vet inte om elekronisk MADRS gör någon skillnad-vi gör gu uppöljning ändå, t,ex, ringer upp. %<br><a href='#emadrs\_not\_in\_dev\_everybody\_too\_many+b'>Back<a><br><br>


\end{document}

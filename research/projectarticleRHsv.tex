\documentclass[12pt,a4paper,oneside]{article}
\newcommand{\latex}{\LaTeX\xspace}
\usepackage{textcomp}
\usepackage{tabularx}
\usepackage[table]{xcolor}% http://ctan.org/pkg/xcolo
\usepackage[latin1]{inputenc}
%\usepackage[swedish]{babel}
\usepackage{/usr/lib/R/share/texmf/tex/latex/Sweave}
\usepackage[math]{iwona}
\usepackage[T1]{fontenc}
%\usepackage[swedish]{babel}
\usepackage[UKenglish]{babel}
\usepackage{graphicx}
\usepackage{hyperref}
\usepackage{url} 
\renewcommand{\oldstylenums}[1]{{\fontfamily{pplj}\selectfont #1}}
%\usepackage[textwidth=11cm]{geometry}
%\newcommand{\blob}{\rule[-.2\unitlength]{2\unitlength}{.5\unitlength}}
%\renewcommand{\_}{\hspace{0.1cm}}
\usepackage{bold-extra}
\usepackage{multirow}
%\urlstyle{same}
%\urlstyle{sf}
\def\mydate{\leavevmode\hbox{\the\year-\twodigits\month-\twodigits\day}}
\def\twodigits#1{\ifnum#1<10 0\fi\the#1}
\usepackage[round,comma]{natbib}
\usepackage{enumitem}
\usepackage{titlesec}
\titleformat*{\section}{\normalsize\bfseries\vspace{0.25cm}}
%\titleformat*{\subsection}{\Large\bfseries}
%\titleformat*{\subsubsection}{\large\bfseries}
%\titleformat*{\paragraph}{\large\bfseries}
%\titleformat*{\subparagraph}{\large\bfseries}

\let\oldcite\cite
\renewcommand*\cite[1]{\textsuperscript{\oldcite{#1}}}

\makeatother
\bibliographystyle{unsrt}
%\bibliographystyle{ieeetr}
%\usepackage{natbib}
\usepackage{cclicenses}
\usepackage{nicefrac}

%\bibliography{test}
%\usepackage[sort, numbers]{natbib}
\begin{document}
\title{
\vspace{-3.9cm}
\resizebox{1\hsize}{!}{{\sc\mydate}\ Projektartikel f{\"o}r {\sc t10} vetenskapligt arbete vid l{\"a}karprogrammet, Ume\r{a} universitet}
\\\vspace{-.4cm}
\resizebox{1\hsize}{!}{\fontfamily{iwona}\selectfont\small Ladda ner svensk version fr\r{a}n: {\textbf\emph{\href{https://github.com/RickardHultgren/emadrs/blob/master/research/projectarticleRHsv.pdf}{\url{https://github.com/RickardHultgren/emadrs/blob/master/research/projectarticleRHsv.pdf}}}}}\\\vspace{-.4cm}
\resizebox{1\hsize}{!}{\fontfamily{iwona}\selectfont\small Download English version from: {\textbf\emph{\href{https://github.com/RickardHultgren/emadrs/blob/master/research/projectarticleRH.pdf}{\url{https://github.com/RickardHultgren/emadrs/blob/master/research/projectarticleRH.pdf}}}}}
\\\vspace{-.4cm}
\resizebox{1\hsize}{!}{{\fontfamily{iwona}\selectfont\small\cc Detta verk av Rickard Hultgren {\"a}r licensierat under en Creative Commons Erk{\"a}nnande {\sc 4.0} Internationell Licens.}}
\\\vspace{-.3cm}
\vspace{-0.5cm}
\fontsize{12pt}{12pt}{\fontfamily{iwona}\selectfont
}}
\date{}
\maketitle
\vspace{-0.9cm}
%\hrule
\hrule
\ \\ 
{\small F{\"o}rste f{\"o}rfattare:} {\small \bf Rickard Hultgren} {\small \it rihu0003@student.umu.se}\vspace{.25cm}\\
 {\small Handledare:} {\small \bf Mikael Sandlund} {\small \it mikael.sandlund@umu.se}\\
 {\small\it Inst f klinisk vetenskap/psykiatri; Ume\r{a} universitet; 901 85 Ume\r{a}}\vspace{.25cm}\\
% {\small Assistant supervisor:} {\small \bf Helj{\"a} Pihkala} {\small \it helj{\"a}.p@hotmail.com}\\
 {\small Bihandledare:} {\small \bf Helj{\"a} Pihkala} {\small \it helja.pihkala@umu.se}\\
 {\small\it Inst f klinisk vetenskap/psykiatri; Ume\r{a} universitet; 901 85 Ume\r{a}}\vspace{.5cm}\\
{\fontsize{15pt}{15pt}{\fontfamily{ptm}\selectfont{Projektets titel:}}\\\fontsize{18pt}{18pt}{\fontfamily{ptm}\selectfont{ Personalens inst{\"a}llning till uppf{\"o}ljning och screening via patientens smartphone, exemplifierad av ett fr{\aa}geformul{\"a}r f{\"o}r sj{\"a}lvbed{\"o}mning av depressionssymptom.}}}\\
\hrule
\ \\
\fontfamily{iwona}\selectfont

\begin{abstract}
\ \\\vspace{-2em}\ \\
%\emph{
\bfseries{
H{\"a}lso- och sjukv{\aa}rden beh{\"o}ver nya, kostnadseffektiva verktyg. Hur skulle h{\"a}lso- och sjukv{\aa}rden p{\aa}verkas om prim{\"a}rv{\aa}rden skulle f{\aa} resultat av fr{\aa}geformul{\"a}r fr{\aa}n patientens smartphone? Intervjuer om detta {\"a}mne har utf{\"o}rts med fokusgrupper som inneh{\aa}ller prim{\"a}rv{\aa}rdspersonal p{\aa} Hagfors v{\aa}rdcentral. Inspelningarna unders{\"o}ktes med kvalitativ inneh{\aa}llsanalys. Projektet visar att digitala fr{\aa}geformul{\"a}r har potentialer vid screening och uppf{\"o}ljning.
}
\end{abstract}

\section*{Bakgrund}
I Sverige uppskattas livstidsprevalensen av depression vara 13,2\% hos m{\"a}n och 25,1\% bland kvinnor\cite{numbers0}. Det finns ett v{\"a}letablerat f{\"o}rh{\aa}llande mellan sj{\"a}lvmord och affektiva sjukdomar\cite{numbers1.1}. Det har uppskattats att 50-80\% av f{\"a}rdiga sj{\"a}lvmord {\"a}r associerade med affektiva sjukdomar\cite{numbers1.1}. Sj{\"a}lvmord {\"a}r den fr{\"a}msta orsaken till d{\"o}dsfall bland m{\"a}n mellan 15 och 44 {\aa}r i Sverige\cite{numbers3.0.1}. {\"A}nd{\aa} uppskattas det att drygt \nicefrac {2} {3} av alla sj{\"a}lvmordsfall nyligen hade varit i kontakt med v{\aa}rden. Endast \nicefrac {1} {3} av alla sj{\"a}lvmordsfall hade kontakt med en psykiatrisk klinik\cite{numbers2}. I m{\aa}nga fall kunde sj{\"a}lvmordet ha f{\"o}rhindrats om adekvata anstr{\"a}ngningar hade gjorts\cite{numbers1}. Riktlinjer f{\"o}r behandling och uppf{\"o}ljning av depression finns, men {\"o}kningen av mentala problem bland unga utg{\"o}r en stor utmaning\cite{guide1, regionjh1}.

Att l{\"o}sa den sv{\aa}ra situationen kr{\"a}ver s{\aa}ledes nya s{\"a}tt att hantera depression. Vissa smartphone-appar har utvecklats f{\"o}r att gynna sjukv{\aa}rden hos deprimerade patienter. Apparna kan kategoriseras i tv{\aa} grupper beroende p{\aa} vilken slutanv{\"a}ndare de {\"a}r avsedda f{\"o}r. Om slutanv{\"a}ndaren {\"a}r en patient hj{\"a}lper appen patienten att uppt{\"a}cka och f{\"o}rst{\aa} symptomen genom en st{\"a}mningsdagbok\cite {app1}. Om appen {\"a}r avsedd att anv{\"a}ndas av v{\aa}rdpersonal, {\"a}r appen uppbyggd kring olika fr{\aa}geformul{\"a}r\cite {app2}. B{\aa}da metoderna kan leda till n{\aa}got b{\"a}ttre resultat f{\"o}r patienten, men genom att fokusera p{\aa} antingen patienten eller personalen ignoreras en nyckelaspekt. F{\"o}r att v{\aa}rden ska kunna hj{\"a}lpa patienten s{\aa} bra och effektiv som m{\"o}jligt, {\"a}r det n{\"o}dv{\"a}ndigt att fokusera p{\aa} kommunikationen mellan b{\aa}da parter.

\section*{Syfte}
F{\"o}r att v{\aa}rden ska kunna ge den deprimerade patienten tillr{\"a}cklig hj{\"a}lp, beh{\"o}ver personalen adekvat information om patienten. Vid unders{\"o}kningar av somatiska patologier utf{\"o}rs vanliga laboratorietester vanligtvis f{\"o}re en tidpunkt. Vad h{\"a}nder om patientens nedst{\"a}mdhet kan m{\"a}tas p{\aa} ett liknande s{\"a}tt f{\"o}re en tid? I syfte att f{\"o}rb{\"a}ttra kommunikationen mellan patient och v{\aa}rdpersonal har en app-prototyp (e{\sc madrs}) f{\"o}r android-smartphones utvecklats av den f{\"o}rsta f{\"o}rfattaren\cite {emadrs1, emadrs2}. Appen best{\aa}r av en {\sc madrs-s} blankett. Resultatet skickas till ett telefonnummer som f{\"o}redras av patienten som ett {\sc sms}-textmeddelande. {\sc Madrs-s} {\"a}r ett verifierat verktyg som vanligtvis anv{\"a}nds f{\"o}r screening och uppf{\"o}ljning av depression\cite {madrs2, madrs3}. Den best{\aa}r av nio fr{\aa}gor. Patienten svarar p{\aa} varje fr{\aa}ga med ett betyg fr{\aa}n noll till sex. Den sammanlagda po{\"a}ngen fr{\aa}n alla fr{\aa}gor kategoriseras enligt f{\"o}ljande: \\

\begin{tabular}{r|l}
{\bf score} & {\bf allvarlighetsgrad} \\
\hline {\it 0--6} & ingen depression \\
{\it 7-19} & mild depression \\
{\it 20--34} & m{\aa}ttlig depression \\
{\it 35--54} & allvarlig depression \\\end{tabular}\vspace{1em}
\\Forsknings{\"a}mnena {\"a}r tre omr{\aa}den, n{\"a}ra f{\"o}rbundna med varandra:
\begin{itemize}
\item[$\alpha$] Vilka f{\"o}rdelar och nackdelar identifieras ur ett professionellt kliniskt perspektiv, med hj{\"a}lp av ett digitalt utv{\"a}rderingsinstrument f{\"o}r depression i screening och uppf{\"o}ljning?
\item[$\beta$] Syftet {\"a}r ocks{\aa} att samla f{\"o}rslag till vidareutveckling av e{\sc madrs}.
\item[$\gamma$] Vilka personalkategorier skulle p{\aa}verkas mest av digitala fr{\aa}geformul{\"a}r?
\end{itemize}

\section*{Material och metod}
%Interviews with at least {\sc 2} focus groups, consisting of about {\sc 5--7} primary care unit employees from different staff categories that are involved in the treatment of depression. The staff categories concerned are primarily physicians, nurses and members from the so-called psycho-social teams (e.g. psychologists)\cite{goal1}. Each group will consist of as few staff categories as possible. This would be beneficial for the interviews since it would minimize the risk of hierarchical group dynamics.
Tv{\aa} fokusgrupper bildades, best{\aa}ende av sju respektive sex anst{\"a}llda inom prim{\"a}rav{\aa}rden fr{\aa}n olika personalkategorier som direkt eller indirekt {\"a}r involverade i behandling av depression hos Hagfors v{\aa}rdcentral. F{\"o}ljande tabell {\"a}r en sammanst{\"a}llning av gruppmedlemmarna:\\\\
\begin{tabular}{p{3em}|p{10em}|l|l}
Grupp & Arbetstitel & Intervju 1 & Intervju 2 \\
\hline
\multirow {5} {*} {A} & Administrat{\"o}rer & 1 & 1 \\
& Sjuksk{\"o}terskor & 3 & 3 \\
& Fotterapeuter & 1 & 0 \\
& L{\"a}kare & 0 & 1 \\
& Psykoterapeuter & 1 & 1 \\
\hline
\multirow {6} {*} {B} & Administrat{\"o}rer & 1 & 1 \\
& Undersk{\"o}terskor & 1 & 2 \\
& Sjuksk{\"o}terskor & 0 & 1 \\
& Fotterapeuter & 0 & 0 \\
& L{\"a}karassistenter & 1 & 1 \\
& Psykoterapeuter & 1 & 1 \\
\hline
\end{tabular}\\\\\\\\
Tv{\aa} 30-minuters intervjuer utf{\"o}rdes med varje grupp. F{\"o}r att f{\aa} en helhetsbild av hur en v{\aa}rdcentral skulle p{\aa}verkas av e{\sc madrs}, intervjuades m{\aa}nga personalkategorier\cite {goal1}. Under intervjuerna diskuterades f{\"o}ljande fr{\aa}gor: \\\\\
Intervju 1%\vspace{-.5em}
\begin{enumerate}[label=1.\arabic*.]
\item {\bf} Vad {\"a}r specifikt, m{\"a}tbart och uppn{\aa}eligt i ditt arbete?
\end{enumerate}
Intervju 2%\vspace{-.5em}
\begin{enumerate}[label=2.\arabic*.]
%\setlength\itemsep{0em}
\item {\bf} Beskriv dina k{\"a}nslor n{\"a}r patientens huvudproblem inte {\"a}r relaterat till depression, men patienten verkar vara i mycket ned{\"a}md?
\item {\bf} Scenarious diskuteras:
\begin{itemize}\vspace{-.5em}
% \Setlength \itemsep {0em}
\item {\bf} Vad h{\"a}nder om e{\sc madrs} bara kan anv{\"a}ndas f{\"or} uppf{\"o}ljning?
\item {\bf} Vad h{\"a}nder om e{\sc madrs} kan anv{\"a}ndas av vem som helst f{\"o}r att skicka dig bed{\"o}mningar av sitt affektiva tillst{\aa}nd?
\item {\bf} Vad h{\"a}nder om resultatet av e{\sc madrs} automatiskt skulle kunna reglera vilka laboratorietester som ska utf{\"o}ras?
\end{itemize}
\end{enumerate}
Intervjuerna analyserades sedan med hj{\"a}lp av kvalitativ inneh{\aa}llsanalys\cite{analysis1}. Fr{\aa}n inspelningarna h{\"a}rleddes och kategoriserades orsakskoder. Arbetet gjordes med hj{\"a}lp av programmeringsspr{\aa}kbiblioteket {\sc rqda}\cite{rqda}.

\section*{Resultat}
{\bf Forskningsomr{\aa}de $\alpha$:} Runt exemplet e{\sc madrs} identifierades f{\"o}ljande potentialer, styrkor och svagheter:
\begin{itemize}
\item E{\sc madrs} kan vara mycket anv{\"a}ndbart f{\"o}r att f{\"o}lja upp patienter som {\"a}r i riskzonen f{\"o}r {\aa}terfall av depression. \\{\it Kodnamn i appendix: {\bfseries emadrs\_in\_dev\_only\_follow\_up}}
\item Det finns ett behov av digitala verktyg med validerade fr{\aa}geformul{\"a}r f{\"o}r ett bredare spektrum av patologier. \\{\it Kodnamn i appendix: {\bfseries emadrs\_in\_dev\_screening\_follow\_up}}
\item Dessa fr{\aa}geformul{\"a}r ska vara kopplade till varandra p{\aa} ett kontrollerat s{\"a}tt. \\{\it Kodnamn i bilaga: {\bfseries emadrs\_in\_dev\_controll}}
\item E{\sc madrs} kan minska administrat{\"o}rens arbetsbelastning \\{\it kodnamn i appendix: {\bfseries emadrs\_already\_ (+)\_ less\_paper\_work}}
\item Viktigt att n{\aa}gon {\"a}r betrodd och ansvarar f{\"o}r de inkomna fr{\aa}geformul{\"a}rens resultat. \\{\it Kodnamn i bilaga: {\bfseries emadrs\_already\_ (+)\_ possebility\_to\_check}}
\item E{\sc madrs} b{\"o}r inte anv{\"a}ndas f{\"o}r att diagnostisera depression. \\{\it Kodnamn i bilaga: {\bfseries emadrs\_not\_in\_dev\_verybody\_diagnostic}}
\item E{\sc madrs} b{\"o}r inte vara m{\"o}jligt att anv{\"a}nda av vem som helst f{\"o}r att skicka resultaten till v{\aa}rdgivaren. \\{\it Kodnamn i appendix: {\bfseries emadrs\_not\_in\_dev\_very\_f{\"o}r m{\aa}nga}}
\end{itemize}
% Det b{\"o}r finnas n{\aa}gon form av fl{\"o}desstyrning f{\"o}r dessa fr{\aa}geformul{\"a}r.
{\bf Forskningsomr{\aa}de $\beta$ och $\gamma$:} I en effektiv organisation skiljer man hanteringsprocessen fr{\aa}n ledarskapet\cite {leader1}. Nedan f{\"o}ljer en tabell som beskriver vilka olika personalkategorier som hanterar eller leder i sitt arbete enligt intervjuerna:\nopagebreak
\begin{table}
\begin{tabularx}{\textwidth}{|X|X|}
\hline
Uppgifter om att kontakta patienten, vilket inneb{\"a}r ledarskap
{\newline \tiny kodning i appindix: {leadership\_contact\_patient}} & {\begin{itemize}
\vspace{-1.5em} \setlength \itemsep {0em}
{\item assistentl{\"a}kare}
{\item undersk{\"o}terska}
{\item fotterapeut}
{\item sjuksk{\"o}terska}
{\item sjuksk{\"o}terska COPD}
{\item sjuksk{\"o}terska DM2}
{\item sjuksk{\"o}terska geriatric}
{\item physician}
{\item psychotherapist}\vspace{-.5em}\end{itemize}}\\
\hline
Uppgifter om att kontakta patienten, vilket inneb{\"a}r hantering
{\newline \tiny kodning i appindix: {management\_contact\_patient}} & {\begin{itemize}
\vspace{-1.5em} \setlength \itemsep{0em}
{\item administrator}
{\item sjuksk{\"o}terska}
{\item psykoterapeut}
\vspace{-.5em}\end{itemize}}\\
\hline
Uppgifter om k{\"a}nsla av empati, vilket inneb{\"a}r ledarskap
{\newline \tiny kodning i appindix: {leadership\_empathy}} & {\begin{itemize}
\vspace{-1.5em} \setlength\itemsep{0em}
{\item administrator}
\vspace{-.5em}\end{itemize}}\\
\hline
Uppgifter att kontakta personalen, vilket inneb{\"a}r ledning
{\newline \tiny kodning i appindix: {management\_contact\_staff}} & {\begin{itemize}
\vspace{-1.5em} \setlength\itemsep{0em}
{\item foot\_therapist}
\vspace{-.5em}\end{itemize}}\\
\hline
Hantera finanser
{\newline \tiny kodning i appindix: {management\_financial}} & {\begin{itemize}
\vspace{-1.5em} \setlength\itemsep{0em}
{\item administrator}
\vspace{-.5em}\end{itemize}}\\
\hline
Uppgifter f{\"o}r att uppr{\"a}tth{\aa}lla patientens h{\"a}lsa, vilket inneb{\"a}r att patienten ska hanteras
{\newline \tiny kodning i appindix: \newline {management\_medical\_practice\_patient\_part}} & {\begin{itemize}
\vspace{-1.5em} \setlength\itemsep{0em}
{\item fotterapeut}
{\item sjuksk{\"o}terska COPD}
{\item sjuksk{\"o}terska DM2}
\vspace{-.5em}\end{itemize}}\\
\hline
Uppgifter f{\"o}r att beh{\aa}lla patientens h{\"a}lsa, vilket inneb{\"a}r att man hanterar andra yrkesverksamma
{\newline \tiny kodning i appindix: {management\_medical\_practice\_staff\_part}} & {\begin{itemize}
\vspace{-1.5em}\setlength\itemsep{0em}
{\item assistentl{\"a}kare}
{\item undersk{\"o}terska}
{\item sjuksk{\"o}terska COPD}
{\item sjuksk{\"o}terska DM2}
{\item sjuksk{\"o}terska geriatric}
{\item physician}
{\item psykoterapeut}
\vspace{-.5em}\end{itemize}}\\
\hline
Uppgifter f{\"o}r hantering av journaler
{\newline \tiny kodning i appindix: {management\_medical\_record}} & {\begin{itemize}
\vspace{-1.5em}\setlength\itemsep{0em}
{\item administrator}\vspace{-.5em}\end{itemize}}\\
\hline
\end{tabularx}
\end{table}
\newpage
Digitala fr{\aa}geformul{\"a}r {\"a}r ett s{\"a}tt att hantera kommunikationen med patienten. Enligt tabellen ovan {\"a}r personalkategorierna som arbetar med den uppgiften: sjuksk{\"o}terskor, administrat{\"o}rer och psykoterapeuter. S{\aa}lunda skulle dessa personalkategorier sannolikt blir mest p{\aa}verkade av anv{\"a}ndningen av digitala fr{\aa}geformul{\"a}r.


\section*{Diskussion}
%yAccording to the Swedish law: ''The goal of the health care is a good health and care on equal terms for the entire population''\cite{law}. 
Under de senaste sju {\aa}ren har landstingets utgifter {\"o}kat med cirka 5\% per {\aa}r. Justerat f{\"o}r inflationen blir det ungef{\"a}r 3\% per {\aa}r\cite{numbers3.1, numbers3.2}. Strategierna inom v{\aa}rden m{\aa}ste f{\"o}r{\"a}ndras. F{\"o}rhoppningsvis kan detta projekt vara ett steg i r{\"a}tt riktning. Resultaten visar nya s{\"a}tt att f{\"o}rb{\"a}ttra kommunikationen mellan sjukv{\aa}rdssystemet och patienten.
F{\"o}r att kunna f{\"o}rutse f{\"o}r{\"a}ndringar i arbetsbelastningen som de nya digitala verktygen kan leda till m{\aa}ste ytterligare vetenskapligt arbete g{\"o}ras. %Detta {\"a}r viktigt eftersom ekonomiska f{\"o}rdelar endast kan uppn{\aa}s om arbetsuppgifterna f{\"o}r{\"a}ndras, eftersom finanserna {\"a}r en spegling av vilket arbete som {\"a}r aktuellt.
%\section*{Planned format and language of the essay}
%The work will be a scientific article in English with the Oxford citation and journal style guidelines\cite{style1}. The cited %article and {\fontfamily{pplj}\selectfont{\LaTeX}} will be used for writing the article\cite{style2}.

%\section*{Timetable}
%The main parts of the project and assigned duration in weeks are listed below: 
%\begin{tabular}{|r|p{2cm}|}
%\hline
%Preparation of the project plan and creation of focus groups. & {{6}}\\
%Collecting data. & {{2}}\\
%Data processing. & {{6}}\\
%Project report. & {{3}}\\
%Preparing for the presentation. & {{2}}\\
%\hline
%\end{tabular}\\

\section*{Etik}
Projektets karakt{\"a}r {\"a}r utvecklingsarbete inom kliniken. D{\"a}rf{\"o}r har projektet godk{\"a}nts ur sekretess och s{\"a}kerhet av verksamhetschefen vid Hagfors v{\aa}rdcentral. Projektet omfattas inte av Etikprövningslagens forskningsdefinition.

\begin{thebibliography}{11}
\bibitem{numbers0}Kendler KS, Gatz M, Gardner CO, Pedersen NL\emph{A Swedish national twin study of lifetime major depression.}; Am J Psychiatry. 2006 Jan; 163(1):109-14.\\\textbf{\emph{\href{https://www.ncbi.nlm.nih.gov/pubmed/16390897/}{\url{https://www.ncbi.nlm.nih.gov/pubmed/16390897/}}}}
\bibitem{numbers1} \emph{Utv{\"a}rdering 2013 -- v\r{a}rd och insatser vid depression, \r{a}ngest och schizofreni. Indikatorer och underlag f{\"o}r bed{\"o}mningar.}; Socialstyrelsen\\\textbf{\emph{\href{http://www.socialstyrelsen.se/publikationer2013/2013-6-7}{\url{http://www.socialstyrelsen.se/publikationer2013/2013-6-7}}}}
\bibitem{numbers1.1} Kasper S1, Schindler S, Neumeister A.; \emph{Risk of suicide in depression and its implication for psychopharmacological treatment.}; Int Clin Psychopharmacol. 1996 Jun;11(2):71-9.\\\textbf{\emph{\href{https://www.ncbi.nlm.nih.gov/pubmed/8803644}{\url{https://www.ncbi.nlm.nih.gov/pubmed/8803644}}}}
\bibitem{numbers2} \emph{Sj{\"a}lvmord i anslutning till v\r{a}rd Socialstyrelsen};\\\textbf{\emph{\href{http://www.socialstyrelsen.se/patientsakerhet/riskomraden/suicid}{\url{http://www.socialstyrelsen.se/patientsakerhet/riskomraden/suicid
}}}}
\bibitem{numbers3.0.1}\emph{Statistics on causes of death 2015 - Socialstyrelsen}; Socialstyrelsen;
\\\textbf{\emph{\href{https://www.socialstyrelsen.se/Lists/Artikelkatalog/Attachments/20291/2016-8-4.pdf}{\url{
https://www.socialstyrelsen.se/Lists/Artikelkatalog/Attachments/20291/2016-8-4.pdf}}}}
\bibitem{numbers3.1} \emph{Resultatr{\"a}kning f{\"o}r landsting \r{a}r 2010--2014}; SCB;\\\textbf{\emph{\href{ http://www.scb.se/hitta-statistik/statistik-efter-amne/offentlig-ekonomi/finanser-for-den-kommunala-sektorn/rakenskapssammandrag-for-kommuner-och-landsting/pong/tabell-och-diagram/kommun--och-landstingssektorn-2014/resultatrakning-for-landsting-ar-20102014/}{\url{http://www.scb.se/hitta-statistik/statistik-efter-amne/offentlig-ekonomi/finanser-for-den-kommunala-sektorn/rakenskapssammandrag-for-kommuner-och-landsting/pong/tabell-och-diagram/kommun--och-landstingssektorn-2014/resultatrakning-for-landsting-ar-20102014/}}}}
\bibitem{numbers3.2} \emph{Resultatr{\"a}kning f{\"o}r landsting \r{a}r 2012--2016}; SCB;\\\textbf{\emph{\href{
http://www.scb.se/hitta-statistik/statistik-efter-amne/offentlig-ekonomi/finanser-for-den-kommunala-sektorn/rakenskapssammandrag-for-kommuner-och-landsting/pong/tabell-och-diagram/kommun--och-landstingssektorn-2016/resultatrakning-for-landsting-ar-2012-2016/}{\url{http://www.scb.se/hitta-statistik/statistik-efter-amne/offentlig-ekonomi/finanser-for-den-kommunala-sektorn/rakenskapssammandrag-for-kommuner-och-landsting/pong/tabell-och-diagram/kommun--och-landstingssektorn-2016/resultatrakning-for-landsting-ar-2012-2016/}}}}
\bibitem{app1}\emph{Appen Uppskatta}; Google Play\\\textbf{\emph{\href https://play.google.com/store/apps/details?id=com.akerlund.uppskattadindag}{\url{{https://play.google.com/store/apps/details?id=com.akerlund.uppskattadindag}}}}
\bibitem{app2}\emph{Appen PsykTools}; Google Play\\\textbf{\emph{\href https://play.google.com/store/apps/details?id=no.sonat.honos}{\url{{https://play.google.com/store/apps/details?id=no.sonat.honos}}}}
%\bibitem{madrs1}Zimmerman, Chelminski, Posternak \emph{The Montgomery Asberg Depression Rating Scale in bipolar II and unipolar out-patients: a 405-patient case study.}; Int Clin Psychopharmacol. 2004 Jan;19(1):1-7; \\\textbf{\emph{\href{http://www.ncbi.nlm.nih.gov/pubmed/15101563}{\url{http://www.ncbi.nlm.nih.gov/pubmed/15101563}}}}
%\bibitem{madrs2}Svanborg, P; \r{A}sberg, M; \emph{A comparison between the Beck Depression Inventory (BDI) and the self-rating version of the Montgomery {\"A}\r{A}sberg Depression Rating Scale (MADRS)}; J. Affective Disorders. 64 (2-3): 203--216. doi:10.1016/S0165-0327(00)00242-1.\\\textbf{\emph{\href{https://www.ncbi.nlm.nih.gov/pubmed/11313087}{\url{https://www.ncbi.nlm.nih.gov/pubmed/11313087}}}}
\bibitem{madrs2}Svanborg, P; \r{A}sberg, M; \emph{A comparison between the Beck Depression Inventory (BDI) and the self-rating version of the Montgomery \r{A}sberg Depression Rating Scale (MADRS)}; J. Affective Disorders. 64 (2-3): 203--216. doi:10.1016/S0165-0327(00)00242-1.\\\textbf{\emph{\href{https://www.ncbi.nlm.nih.gov/pubmed/11313087}{\url{https://www.ncbi.nlm.nih.gov/pubmed/11313087}}}}
\bibitem{madrs3}\emph{Tolkning av MADRS-S}; Region J{\"o}nk{\"o}pings l{\"a}n\\\textbf{\emph{\href{http://plus.rjl.se/info_files/infosida39803/madrs_s_tolkning.pdf}{\url{http://plus.rjl.se/info_files/infosida39803/madrs_s_tolkning.pdf}}}}
\bibitem{emadrs1}Rickard Hultgren; \emph{eMADRS source code}; github.com; \\\textbf{\emph{\href{https://github.com/RickardHultgren/emadrs}{\url{https://github.com/RickardHultgren/emadrs}}}}
\bibitem{emadrs2}Rickard Hultgren; \emph{eMADRS compiled}; play.google.com; \\\textbf{\emph{\href{https://play.google.com/store/apps/details?id=rickardverner.hultgren.emadrs}{\url{https://play.google.com/store/apps/details?id=rickardverner.hultgren.emadrs}}}}
%\bibitem{goal0}Birgitta Lindelius; \emph{{\"O}ppna j{\"a}mf{\"o}relser och utv{\"a}rdering 2010 -- Psykiatrisk v\r{a}rd}; socialstyrelsen.se; \\\textbf{\emph{\href{http://www.socialstyrelsen.se/publikationer2010/2010-6-6}{\url{http://www.socialstyrelsen.se/publikationer2010/2010-6-6}}}}
\bibitem{goal1}Tracy R.G. Gladstone, William R. Beardslee, Erin E. O'Connor; \emph{The Prevention of Adolescent Depression}; Psychiatr Clin North Am. 2011 Mar; 34(1): 35--52. \\\textbf{\emph{\href{https://www.ncbi.nlm.nih.gov/pmc/articles/PMC3072710/}{\url{https://www.ncbi.nlm.nih.gov/pmc/articles/PMC3072710/}}}}
%\bibitem{goal2}Maryann Davis, Michael T. Abrams, Lawrence S. Wissow, Eric P. Slade; \emph{Identifying young adults at risk of Medicaid enrollment lapses after inpatient mental health treatment}; Psychiatr Serv. 2014 Apr 1; 65(4): 461--468.doi:  10.1176/appi.ps.201300199 \\\textbf{\emph{\href{https://www.ncbi.nlm.nih.gov/pmc/articles/PMC3972275/}{\url{https://www.ncbi.nlm.nih.gov/pmc/articles/PMC3972275/}}}}
\bibitem{guide1}Riitta Sorsa; \emph{Nationella riktlinjer -- M\r{a}lniv\r{a}er -- V\r{a}rd vid depression och \r{a}ngestsyndrom -- M\r{a}lniv\r{a}er f{\"o}r indikatorer}; socialstyrelsen.se december 2017\\\textbf{\emph{\href{http://www.socialstyrelsen.se/publikationer2017/2017-12-1}{\url{http://www.socialstyrelsen.se/publikationer2017/2017-12-1}}}}
\bibitem{regionjh1}Majvor Enstr{\"o}m; \emph{Granskning av Psykiatrin 2014 Region J{\"a}mtland-H{\"a}rjedalen}\\\textbf{\emph{\href{https://www.regionjh.se/download/18.61342ea415bcfb51720c5fd7}{\url{https://www.regionjh.se/download/18.61342ea415bcfb51720c5fd7}}}}
%\bibitem{style0}Vakhtang Tchantchaleishvili, Jan D Schmitto; \emph{Preparing a scientific manuscript in Linux: Today's possibilities and limitations.}; BMC Res Notes. 2011; 4: 434. \\\textbf{\emph{\href{https://www.ncbi.nlm.nih.gov/pmc/articles/PMC3227619/}{\url{https://www.ncbi.nlm.nih.gov/pmc/articles/PMC3227619/}}}}
%\bibitem{style1}\emph{Style Guide for Authors} \\\textbf{\emph{\href{https://academic.oup.com/cdj/pages/Style_Guide}{\url{https://academic.oup.com/cdj/pages/Style_Guide}}}}
%\bibitem{style2}Barbara J. Hoogenboom, Robert C. Manske; \emph{How to write a scientific article.} Int J Sports Phys Ther. 2012 Oct; 7(5): 512--517. \\\textbf{\emph{\href{https://www.ncbi.nlm.nih.gov/pmc/articles/PMC3474301/}{\url{https://www.ncbi.nlm.nih.gov/pmc/articles/PMC3474301/}}}}
%\bibitem{style3}\emph{Latex Instructions - Elsevier}; Elsevier 2018;\\\textbf{\emph{\href{https://www.elsevier.com/authors/author-schemas/latex-instructions}{\url{https://www.elsevier.com/authors/author-schemas/latex-instructions}}}}
\bibitem{leader1}Kotter JP \emph{What leaders really do.} Harvard Business Review 1990
%\bibitem{ethics1}Will JF \emph{A brief historical and theoretical perspective on patient autonomy and medical decision making: Part I: The beneficence model.} Chest. 2011 Mar;139(3):669-673. doi: 10.1378/chest.10-2532. \\\textbf{\emph{\href{https://www.ncbi.nlm.nih.gov/pubmed/21362653}{\url{https://www.ncbi.nlm.nih.gov/pubmed/21362653}}}}

\bibitem{analysis1}Granheim, Lundman \emph{Qualitative content analysis in nursing research concepts, procedures and measures to achieve trustworthiness.} Nurse Educ Today. 2004 Feb;24(2):105-12 \\\textbf{\emph{\href{https://www.ncbi.nlm.nih.gov/pubmed/14769454}{\url{https://www.ncbi.nlm.nih.gov/pubmed/14769454}}}}

%\bibitem{law}H{\"a}lso- och sjukv\r{a}rdslag (1982:763)\\\textbf{\emph{\href{http://www.notisum.se/rnp/sls/lag/19820763.htm}{\url{http://www.notisum.se/rnp/sls/lag/19820763.htm}}}}

\bibitem{rqda}Ronggui Huang; \\\textbf{\emph{\href{http://rqda.r-forge.r-project.org/}{\url{http://rqda.r-forge.r-project.org/}}}}

\end{thebibliography} 
 
\end{document}

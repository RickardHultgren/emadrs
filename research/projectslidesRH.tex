\documentclass[english]{beamer}
\usepackage[utf8]{inputenc}
\usepackage{url}
\usepackage[english]{babel}
\usepackage[T1]{fontenc}
\usepackage[math]{iwona}
\setcounter{tocdepth}{1}

%\date{October 3--4, 2009}

\usetheme{Boadilla}
\useoutertheme{infolines}
\setbeamerfont{section in head/foot}{family=\sffamily}
\setbeamerfont{subsection in head/foot}{family=\sffamily}
\setbeamerfont{title in head/foot}{family=\sffamily}
\setbeamerfont{author in head/foot}{family=\sffamily}
\setbeamerfont{date in head/foot}{family=\sffamily}
\setbeamerfont{institute in head/foot}{family=\sffamily, size=\tiny}
\setbeamerfont{block title}{family=\sffamily}
\setbeamerfont{block title alerted}{family=\sffamily}
\setbeamerfont{block title example}{family=\sffamily}
%\renewcommand{\familydefault}{\rmdefault}

\begin{document}

\title{Personalens inställning till uppföljning och screening via patientens smartphone, exemplifierad av ett frågeformulär för självbedömning av depressionssymptom.}
%\author[ABC]{ABC\\[5mm]{\footnotesize \textbf{Supervisors:}\\Prof. QWERTY\\Prof. DEF GHI\\Prof. Jklmno Pqrst\\}}
\author[]{Första författare: Rickard Hultgren\newline Handledare: Mikael Sandlund\newline Bihandledare: Heljä Pihkala}
%\url{http://slackware.com/~alien/}\\
%\texttt{alien@slackware.com}

\institute{\\
Umeå universitet\\
%\url{http://umu.se/}
}

\begin{frame}
	\titlepage
\end{frame}


\section{Overview}
\begin{frame}
	\frametitle{Statistik i psykiatri}

	\begin{itemize}
	\item 50--80\% av begångna suicid är associerade med affektiva sjukdomar.
	\item Suicid är den främsta orsaken till dödsfall bland män mellan 15 och 44 år i Sverige.
	\item 2/3 av alla suicid nyligen hade varit i kontakt med vården.
	\end{itemize}

	\pause

	\textit{Slutsats:\\Patientens affektiva förstås inte alltid av läkaren.}\\\ \\

\pause

Läkarens fel?\\
Patientens fel?\\
Dåliga verktyg för kommunikation?
\end{frame}

\subsection{}
\begin{frame}
\ \\ 
eMADRS är en app för android som består av ett digitalt MADRS-S formulär. Resultatet skickas via SMS till angivet nummer.\\\ \\

Att utläsa från MADRS-S resultat:
	\begin{itemize}
	\item allvarighetgraden av symptombilden.
	\item MADRS-S ger ej diagnos.
	\end{itemize}
Källkod:\\
\href{https://github.com/RickardHultgren/emadrs}{\url{https://github.com/RickardHultgren/emadrs}}\\
Nedladdning:\\
\href{https://play.google.com/store/apps/details?id=rickardverner.hultgren.emadrs}{\url{https://play.google.com/store/apps/details?id=rickardverner.hultgren.emadrs}}

\end{frame}

\subsection{}
\begin{frame}
\includegraphics[scale=0.4]{emadrs1.png}
\end{frame}

\begin{frame}
\includegraphics[scale=0.4]{emadrs2.png}
\end{frame}

\begin{frame}
\includegraphics[scale=0.4]{emadrs3.png}
\end{frame}

\begin{frame}
\includegraphics[scale=0.4]{emadrs3.png}
\end{frame}


\begin{frame}
\frametitle{Frågeställningar}
	\begin{itemize}
	\item Vilka fördelar och nackdelar identifieras ur ett professionellt kliniskt perspektiv, med att använda ett digitalt utvärderingsinstrument för depression i screening och uppföljning?
	\item Förslag till vidareutveckling av emadrs.
	\item Vilka personalkategorier skulle påverkas mest av digitala frågeformulär?
	\end{itemize}
\end{frame}

\begin{frame}
\frametitle{Metod}
2 intervjuer av 2 fokusgrupper bestående av olika personalkategorier.\\\ \\
Intervju {\sc 1}\vspace{-.33em}
\begin{enumerate}
\item {\bf} Vad {\"a}r specifikt, m{\"a}tbart och uppn{\aa}eligt i ditt arbete?
\end{enumerate}
Intervju {\sc 2}\vspace{-.33em}
\begin{enumerate}
%\setlength\itemsep{0em}
\item {\bf} Beskriv hur du upplever din arbetssituation n{\"a}r patientens huvudproblem inte {\"a}r relaterat till depression, men patienten verkar vara mycket nedst{\"a}md?\vspace{-.33em}
\item {\bf} Scenarier diskuteras:\vspace{-.33em}
\begin{itemize}\vspace{-.33em}
% \Setlength \itemsep {0em}
\item {\bf} Vad h{\"a}nder om e{\sc madrs} bara kan anv{\"a}ndas f{\"or} uppf{\"o}ljning?\vspace{-.33em}
\item {\bf} Vad h{\"a}nder om e{\sc madrs} kan anv{\"a}ndas av vem som helst f{\"o}r att skicka dig bed{\"o}mningar av sitt affektiva tillst{\aa}nd?\vspace{-.33em}
\item {\bf} Vad h{\"a}nder om resultatet av e{\sc madrs} automatiskt skulle kunna reglera vilka laboratorietester som ska utf{\"o}ras?
\end{itemize}
\end{enumerate}
\end{frame}

\begin{frame}
\frametitle{Resultat}
Angående eMADRS
\begin{itemize}
\item Emadrs kan vara mycket användbart för att följa upp patienter som är i riskzonen för återfall av depression.
\item Emadrs bör inte användas för att diagnostisera depression.
\item Emadrs bör inte vara möjligt att använda av vem som helst för att skicka resultaten till vårdgivaren. 
\item Emadrs kan minska administratörens arbetsbelastning. 
\end{itemize}
\end{frame}

\begin{frame}
\frametitle{Resultat}
Angående digitala frågeformulär
\begin{itemize}
\item Det finns ett behov av digitala verktyg med validerade frågeformulär för ett bredare spektrum av patologier.
\item  Dessa frågeformulär ska vara kopplade till varandra på ett kontrollerat sätt. 
\item  Viktigt att någon ansvarar för, och är betrodd att hantera de inkomna frågeformulärens resultat. 
\end{itemize}
\end{frame}

\begin{frame}
\frametitle{Resultat}
Personalkategorier som sannolikt blir mest påverkade av användningen av digitala frågeformulär:
\begin{itemize}\vspace{-.33em}
% \Setlength \itemsep {0em}
\item Sjuksköterskor
\item Administratörer
\item Samltalsterapeuter

\end{itemize}
\end{frame}

%\begin{frame}
%\frametitle{Betydelse}
%\begin{itemize}\vspace{-.33em}
%\item Digitala frågeformulär kan förhoppningsvis vara ett stöd i kommunikationen med patienten.
%\item Genom att minska den administrativa bördan skulle vårdpersonal kunna fokusera mer på samspelet med patienten.
%\end{itemize}
%\end{frame}

\begin{frame}
\frametitle{Betydelse}
Potentiella fördelar med digitala formulär på mobiltelefon:
\begin{itemize}\vspace{-.33em}
% \Setlength \itemsep {0em}
\item Förbättrar vård
\item Minska administrativ börda
\item Minska kostnader för landstingen
\end{itemize}
\end{frame}

\end{document}
